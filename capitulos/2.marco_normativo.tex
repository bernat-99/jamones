\chapter{MARCO NORMATIVO}\label{ch:marco}

\section{Normativa Española y Europea Aplicable}\label{sec:marco-normativa}
\subsection{Reglamento (UE) 2024/1781 (ESPR) y Pasaporte Digital de Producto (DPP)}
En junio de 2024 la UE adoptó el Reglamento (UE) 2024/1781 sobre ecodiseño de productos sostenibles, conocido por sus siglas en inglés ESPR (\textit{Ecodesign for Sustainable Products Regulation}). Este reglamento (de aplicación directa en todos los Estados Miembro) establece un nuevo marco legal para impulsar productos más sostenibles en el mercado europeo. A diferencia de la antigua directiva de ecodiseño (centrada solo en productos energéticos), el ESPR abarca prácticamente todos los bienes físicos (electrónica, textiles, muebles, materiales de construcción, baterías, etc.), y tiene como objetivos principales reducir los impactos medioambientales y sociales de los productos, promover la economía circular (mejor reparabilidad, reutilización y reciclaje), minimizar residuos y emisiones, y mejorar la transparencia para consumidores e industrias mediante pasaportes digitales de producto. \cite{ReglamentoUE2024a}

El elemento clave del ESPR es el Pasaporte Digital de Producto (DPP): según el artículo 9 del reglamento, el DPP será obligatorio para los productos que determine la Comisión y servirá como herramienta para recopilar y compartir información detallada del producto a lo largo de toda su cadena de valor. En otras palabras, cada producto deberá ir acompañado de un <<perfil digital>> accesible mediante un identificador (por ejemplo, un código QR escaneable) que contendrá datos como: composición de materiales (incluyendo contenido reciclado o sustancias peligrosas), huella ambiental (carbono, agua), información sobre reparabilidad y disponibilidad de repuestos, vida útil estimada y opciones de reciclaje al final de su vida, uso de energía o recursos, e incluso certificados o documentos de cumplimiento normativo. El objetivo es que todos los agentes (fabricantes, distribuidores, consumidores, recicladores, autoridades) tengan acceso a información verificada y actualizada del producto, cada uno según su nivel de autorización. Por ejemplo, un consumidor podría escanear el DPP de un jamón ibérico y consultar datos de sostenibilidad (procedencia de la dehesa, huella de carbono, certificaciones de bienestar animal, etc.), mientras que un inspector podría verificar información de trazabilidad o cumplimiento de estándares.

Cabe destacar que la implantación del DPP será progresiva: el ESPR entró en vigor en julio de 2024, pero los requisitos del pasaporte digital se introducirán gradualmente entre 2025 y 2027, mediante actos delegados que concretarán las obligaciones para cada grupo de productos. Los primeros sectores prioritarios (textil, electrónica, baterías, etc.) servirán de piloto, y se espera que con el tiempo se amplíe a más productos en función de su impacto ambiental y volumen de mercado. Aunque actualmente los alimentos no aparecen entre los sectores iniciales, la visión a largo plazo de la UE es cubrir <<<la gama más amplia posible de productos>>, por lo que no se descarta que en el futuro también productos agroalimentarios de calidad como el jamón ibérico deban incorporar un DPP para informar sobre su sostenibilidad y trazabilidad. \cite{ESPREnergyLabelling}

Desde el punto de vista del diseño del DPP, el ESPR insiste en un enfoque descentralizado y en la protección de los datos. En sus considerandos, el reglamento señala expresamente que el pasaporte digital de producto debe basarse en <<un sistema de datos descentralizado, establecido y gestionado por los operadores económicos>>, en aras de la flexibilidad, innovación y alineación con las prácticas de mercado. Esto implica que no habrá una única base de datos central controlada por la administración para todos los DPP; más bien cada fabricante u operador mantendrá los datos de sus productos de forma interoperable, mientras que la Comisión creará un registro central únicamente de los identificadores únicos de pasaporte para facilitar la supervisión por las autoridades. En paralelo, el ESPR establece la creación de un portal web público de la UE donde consumidores y partes interesadas puedan consultar y comparar datos de los pasaportes digitales, garantizando así la transparencia. Todo ello deberá cumplirse respetando la normativa de protección de datos personales: cualquier tratamiento de datos personales en el contexto del DPP debe observar el RGPD (Reglamento (UE) 2016/679), incorporando los principios de protección de datos desde el diseño y por defecto en la arquitectura del sistema. De hecho, el reglamento aclara que no debe almacenarse información personal de clientes en el pasaporte digital del producto. En definitiva, el nuevo marco legal europeo (ESPR) conecta sostenibilidad, trazabilidad digital y economía circular, y proporciona el fundamento normativo para desarrollar un Pasaporte Digital de Producto aplicado incluso a productos tradicionales como el jamón ibérico.

\subsection{REACH y <<sustancias preocupantes>> en el DPP}\label{sec:marco-reach}
El Reglamento REACH establece un marco para el registro, evaluación, autorización y restricción de sustancias químicas en la UE, incluyendo la lista de <<sustancias extremadamente preocupantes>> (SVHC). % CITA NECESARIA AQUÍ – Añadir fuente adecuada (REACH/REGL. 1907/2006, SVHC).
El ESPR prevé que, cuando aplique, el DPP declare la presencia de sustancias preocupantes para facilitar su gestión a lo largo del ciclo de vida. % CITA NECESARIA AQUÍ – Añadir fuente adecuada (REACH/REGL. 1907/2006, SVHC).
En productos alimentarios como el jamón ibérico, la relevancia de REACH es indirecta y se vincula principalmente a materiales en contacto con alimentos (envases, tintas, adhesivos) más que al jamón en sí. El DPP propuesto contempla campos para declarar la conformidad de estos materiales, alineado con el principio de <<seguridad por diseño>>.

\subsection{Trazabilidad Alimentaria y la Norma de Calidad del Ibérico (Reg. 178/2002, RD 4/2014, DOP)}\label{sec:marco-calidad}
En el ámbito alimentario, la trazabilidad es un requisito legal consolidado en la UE desde hace años (Reglamento (CE) 178/2002, art. 18). Cada operador debe poder identificar y registrar el origen y destino de los alimentos (<<una etapa atrás y una adelante>> en la cadena) para permitir retiradas rápidas en caso de alertas sanitarias. En el caso del jamón ibérico, además de esta trazabilidad general de seguridad alimentaria, existe un marco normativo específico de calidad que refuerza el seguimiento del producto desde la crianza del animal hasta la etiqueta final.

Este marco normativo es la Norma de Calidad del Ibérico, aprobada por Real Decreto 4/2014, que establece las características de calidad que deben cumplir la carne, jamones, paletas y lomos ibéricos. Esta norma introdujo mejoras importantes en etiquetado y trazabilidad, fijando requisitos más estrictos para el uso del término <<ibérico>> con el fin de proteger al consumidor y los atributos diferenciadores del producto. En concreto, el Real Decreto 4/2014 clasifica los productos por tipo de alimentación y pureza de raza (<<100\% ibérico de bellota>>, <<cebo de campo>>, etc.) con precintos de distinto color en cada pieza, y obliga a un etiquetado detallado indicando la categoría. Asimismo, impone controles oficiales e independientes (autoridades autonómicas, entidades certificadoras acreditadas por ENAC[Entidad Nacional de Acreditación], y Consejos Reguladores en productos DOP) para verificar el cumplimiento de la norma. Gracias a esta regulación, hoy cada jamón o paleta ibérica lleva un código único en su precinto que permite rastrear su origen y proceso de producción. Por ejemplo, el código enlaza con datos como la explotación ganadera de nacimiento, la alimentación recibida (bellota y pastos o pienso), fecha de sacrificio, registro del secadero y bodega donde se curó, y fecha de finalización de curación, entre otros.

Esta trazabilidad completa <<de la granja a la mesa>> queda registrada en un sistema informático centralizado del sector (Sistema ITACA), proporcionando garantías de autenticidad al consumidor. Adicionalmente, existen cuatro Denominaciones de Origen Protegidas (DOP) de jamón ibérico (DOP Jabugo, DOP Dehesa de Extremadura, DOP Los Pedroches y DOP Guijuelo), cada una con su propio pliego de condiciones geográficas y de calidad adicionales a la Norma de Calidad. Las DOP, reguladas por el Reglamento (UE) 1151/2012, vinculan el producto a su territorio y método tradicional de elaboración, aportando un nivel extra de trazabilidad y prestigio. Solo los jamones criados, elaborados y curados en la zona geográfica delimitada pueden llevar el sello DOP, y cada Consejo Regulador de DOP gestiona y verifica la trazabilidad e identidad de las piezas amparadas.

Con todo ello, un Pasaporte Digital para el jamón ibérico deberá conciliar ambos planos: cumplir con las normas obligatorias actuales (trazabilidad 178/2002, etiquetado 1169/2011, calidad del ibérico RD 4/2014) al tiempo que incorpora información ampliada de forma voluntaria (huella ambiental, historia de alimentación del animal, etc.), anticipándose a futuras exigencias regulatorias en materia de sostenibilidad \cite{BOEesDOUEL200280201Reglamento}\cite{BOEA2014318RealDecreto}\cite{ministeriodeagriculturapescayalimentacionespanaPliegoCondicionesDenominacion2022}.

\subsection{Pacto Verde Europeo y Estrategias Conexas}\label{sec:marco-pactoverde}
En línea con las estrategias europeas (<<de la Granja a la Mesa>>, biodiversidad, clima, etc.), el modelo extensivo del ibérico en la dehesa presenta ventajas sostenibles: promueve el bienestar animal (cría al aire libre con alimentación natural) y contribuye al desarrollo rural. El sector ibérico actúa como motor socioeconómico en zonas rurales, fijando población y generando empleo local, en concordancia con los Objetivos de Desarrollo Sostenible de la ONU. Todas estas iniciativas sitúan al ibérico en sintonía con las políticas del \textit{European Green Deal}, que busca una Europa climáticamente neutra, circular y justa. La propia Comisión Europea, en el Pacto Verde Europeo de 2019, recalcó la necesidad de transformar la economía hacia la neutralidad climática al 2050 y destacó que los productos sostenibles (incluidos los agroalimentarios) desempeñan un papel esencial en esa transición.

El Pacto Verde impulsa transparencia, eficiencia de recursos y digitalización del ciclo de vida de productos. La estrategia <<de la Granja a la Mesa>> promueve sistemas alimentarios más sostenibles y trazables; la PAC (Política Agraria Común) 2023–2027 incentiva prácticas ambientales y bienestar animal \cite{FarmForkStrategy}. El DPP propuesto se alinea con estos vectores al ofrecer datos verificables sobre procesos (curado, mermas, condiciones ambientales) y certificaciones (bienestar animal), facilitando comunicación responsable hacia mercados internacionales.

\subsection{RGPD y Protección de Datos desde el Diseño}\label{sec:marco-rgpd}
Por otro lado, al diseñar un sistema digital como el DPP, es imprescindible considerar la legislación de protección de datos personales. El Pasaporte Digital de un alimento como el jamón ibérico podría incluir datos referentes a productores (nombre o ubicación de una finca, que a veces puede ser una persona física) u otros datos sensibles a nivel comercial.

El Reglamento General de Protección de Datos (RGPD, UE 2016/679) exige que cualquier tratamiento de datos personales respete principios como minimización (solo recoger los datos necesarios), limitación de la finalidad y seguridad y confidencialidad (art. 5.1 RGPD). Además, el RGPD consagra en su artículo 25 la <<protección de datos desde el diseño y por defecto>>, lo que significa que desde la concepción misma del DPP deben integrarse medidas técnicas y organizativas que garanticen la privacidad (ontrol de accesos, seudonimización, plazos de retención limitados, etc.). El Reglamento ESPR refuerza explícitamente este punto, indicando que el diseño técnico y funcionamiento de los pasaportes digitales de producto deben trasladar los datos de forma segura y respetando las normas de privacidad, prestando especial atención a <<los principios de protección de datos desde el diseño y por defecto>>.

De hecho, se prohíbe almacenar datos personales de los clientes en el pasaporte digital del producto, ya que la finalidad del DPP es la información del producto, no del comprador. Por tanto, en la implementación del DPP para el jamón ibérico se deberá asegurar que cualquier dato personal (por ejemplo, información de contacto de un ganadero si es persona física, usuarios que consulten el pasaporte, o veterinarios oficiales) esté adecuadamente protegido bajo el RGPD: posiblemente recurriendo a anonimización o seudonimización de datos de operadores, obteniendo consentimientos cuando proceda, y garantizando derechos de acceso y rectificación.

En resumen, el marco normativo que rodea al DPP abarca no solo las regulaciones sectoriales de trazabilidad y calidad alimentaria (que proveerán muchos datos al pasaporte), sino también la normativa horizontal de privacidad, que impondrá requisitos de diseño \textit{privacy-by-design} en nuestra solución tecnológica.

\subsection{Sistemas Sectoriales Existentes (ASICI-ÍTACA) y Complementaridad con el DPP}
La Norma de Calidad del Ibérico, establecida por el Real Decreto 4/2014, introdujo el uso obligatorio de precintos inviolables de distinto color (negro, rojo, verde o blanco) para identificar cada jamón y paleta según su categoría comercial \cite{BOEA2014318RealDecreto}. Estos precintos llevan un código único correlacionado con la canal de origen y son asignados bajo la supervisión de la Asociación Interprofesional del Cerdo Ibérico (ASICI), a la cual el Real Decreto encomienda funciones de control sobre su correcta colocación y registro % CITA NECESARIA AQUÍ – Añadir fuente oficial de la Orden AAA/1549/2014.
Para garantizar el cumplimiento de estas obligaciones de trazabilidad y calidad, en 2015 se puso en marcha el sistema informático ÍTACA mediante una extensión de norma aprobada por el Ministerio de Agricultura (Orden AAA/1549/2014) \cite{SistemaITACAAsociacion}.

ÍTACA constituye hoy la plataforma sectorial integral de identificación, trazabilidad y calidad del Ibérico, que recopila de forma unificada la información de todos los operadores (ganaderos, mataderos, industrias) involucrados en la cadena productiva \cite{AppIbericoAsociacion}. A través de este sistema, por ejemplo, los ganaderos gestionan la identificación individual de los animales mediante crotales; los mataderos registran el pesaje de las canales e identifican con precintos oficiales los jamones y paletas aptos de cada lote (según exige el artículo 9 del RD 4/2014) \cite{PrecintosCategoriasAsociacion}; y las industrias anotan el movimiento, transformación y reposición de precintos de las piezas durante su curación y distribución. De este modo se logra una trazabilidad completa <<de la granja a la mesa>>, quedando todos los datos respaldados en la base de datos común de ÍTACA. Desde su implantación, el sistema se ha consolidado como fuente de información en tiempo real para el sector; entre 2015 y 2020 acumuló más de 19 millones de páginas consultadas y 234 mil usuarios registrados \cite{ConsumoRecuerdaQue2022}, reflejando su adopción generalizada y la transparencia que brinda al consumidor.

La propuesta de Pasaporte Digital de Producto (DPP) se plantea como complementaria a ÍTACA, aprovechando la infraestructura existente sin duplicarla. En concreto, reutiliza los identificadores únicos ya asignados en ÍTACA a animales y productos, y agrega eventos de trazabilidad adicionales siguiendo el estándar EPCIS para enriquecer la historia del producto. Además, proporciona un mecanismo de verificación pública \textit{tamper-evident} mediante anclajes criptográficos periódicos de esos datos en la red distribuida IOTA, añadiendo así una capa de transparencia y confianza accesible a cualquier interesado. Un modelo plausible de despliegue sería una gobernanza liderada por la entidad interprofesional (ASICI), que ofrezca servicios comunes de generación y resolución de identificadores digitales, habilitando a los operadores industriales a integrar el DPP en sus sistemas con cambios mínimos. Esto aseguraría que el DPP refuerza la trazabilidad y confianza en el sector ibérico de forma alineada con la normativa vigente, actuando como un complemento que potencia la utilidad de ÍTACA en lugar de reemplazarlo.