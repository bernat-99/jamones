\chapter{DISEÑO DEL DPP}\label{ch:diseño}

\section{Requisitos Funcionales y No Funcionales del DPP para Jamón Ibérico}\label{sec:diseño-requisitos}

\section{Definición del Modelo de Datos: Campos Mínimos y su alineación con GS1/EPCIS}\label{sec:diseño-definicion}
\subsection{Estándares GS1 Europe (GTIN, EPCIS, Digital Link) en el marco DPP}
La implementación práctica del Pasaporte Digital de Producto a escala europea requiere interoperabilidad y estándares comunes. En este contexto, GS1 Europe – organización que gestiona estándares globales de identificación y trazabilidad – juega un papel fundamental

La UE está alineando el DPP con los estándares GS1 más difundidos, como el GTIN (Global Trade Item Number) para la identificación única de productos, EPCIS (Electronic Product Code Information Services) para compartir eventos de la cadena de suministro, y GS1 Digital Link para vincular identificadores a recursos web \cite{noauthor_communication_2025}. En la práctica, esto significa que cada producto con DPP podría identificarse por su GTIN (código de artículo global, base de los códigos de barras), posiblemente extendido con número de lote o serie para distinguir niveles de granularidad (modelo, lote o ítem individual) \cite{gs1_in_europe_gs1-standards-enabling-dpp_2024}. GS1 aboga por usar un URI GS1 Digital Link (un código web incorporando el GTIN y otros datos) impreso en un código QR 2D en el producto, de forma que al escanearlo se acceda al pasaporte digital \cite{noauthor_communication_2025}\cite{gs1_in_europe_gs1-standards-enabling-dpp_2024}. Esto aprovecha la enorme infraestructura ya existente de códigos de barras y aseguraría compatibilidad global. Además, GS1 promueve el uso de identificadores únicos de empresas (GLN) y localizaciones, así como vocabularios estándar para datos de producto, garantizando que la información del DPP sea comprendida y compartida de igual forma en todos los países de la UE \cite{noauthor_communication_2025}.

Un ejemplo es el estándar GS1 EPCIS 2.0, recientemente actualizado, que permite registrar eventos como producción, transporte, recepción, etc., en un formato interoperable; estos eventos podrían alimentarse al pasaporte digital para reflejar la historia logística del jamón ibérico. En cuanto al contenido del DPP, GS1 recomienda incluir el GTIN como identificador principal (no reutilizable una vez asignado a un producto) y combinarlo con datos de lote/serie según se requiera para trazar unidades específicas \cite{gs1_in_europe_gs1-standards-enabling-dpp_2024}. GS1 Europe ha desarrollado incluso una arquitectura de referencia para DPP, destacando que los códigos 2D (como DataMatrix o QR) codificados con sintaxis GS1 Digital Link URI pueden conectar a los consumidores con la información en línea del pasaporte, a la vez que sirven como identificadores robustos en bases de datos y sistemas B2B \cite{gs1_in_europe_gs1-standards-enabling-dpp_2024}.

La colaboración UE-GS1 busca que el Pasaporte Digital de Producto no reinvente la rueda en identificación y trazabilidad, sino que aproveche estándares internacionales ya probados, asegurando interoperabilidad, escalabilidad y adopción masiva. Esto también facilita que iniciativas voluntarias existentes en el sector alimentario (ej. uso de códigos GS1 DataMatrix en etiquetas de alimentos frescos) se integren en un futuro esquema obligatorio de DPP.

\subsection{GTIN/GLN como identificadores}
\subsection{EPCIS 2.0 (eventos y trazabilidad)}
\subsection{GS1 Digital Link (URI/QR 2D)}

\section{Arquitectura Propuesta: sensores IoT, base de Datos Off-Chain, Nodos de IOTA, API de Ingestión y Front-End}\label{sec:diseño-arquitectura}
\subsection{Viabilidad de la arquitectura IoT propuesta (nodos ligeros, Raspberry Pi, testnet, Hornet):}
Para el caso concreto propuesto (Blockchain + IoT para jamón ibérico), se contempla una arquitectura con dispositivos IoT en campo y posiblemente nodos ligeros corriendo en hardware accesible (como Raspberry Pi). IOTA Hornet, el software de nodos para la red IOTA, se ha conseguido hacer funcionar eficientemente en una Raspberry Pi 4 \cite{noauthor_iotaledgerhornet_nodate}. Hornet consume recursos muy reducidos (por diseño en Go), alineándose con escenarios IoT de recursos limitados \cite{pullo_integrating_2024}. Otro caso de uso demuestra que un RPi 4 puede servir simultáneamente como nodo IOTA además de realizar otras tareas, poniendo de manifiesto su ligereza \cite{noauthor_iotaledgerhornet_nodate}. En el prototipo, se podría montar un pequeño cluster de Raspberry Pis actuando como nodos de una red de prueba (testnet) IOTA o incluso una red privada (devnet) para simular la arquitectura. IOTA permite desplegar “private Tangle” fácilmente con coordinador propio, lo cual es útil para fines de desarrollo.

Alternativamente, se podría utilizar la IOTA Shimmer testnet (la red de pruebas pública de IOTA) para publicar transacciones sin coste y con confirmaciones rápidas. Otro componente mencionado es Hornet + Hornet (con nodo Hornet y herramientas Hornet), que se refiere al software Hornet ya discutido. Montar nodos IOTA en RPi con Docker es relativamente sencillo – el \textit{MDPI study} sobre IOTA en agricultura menciona el uso de contenedores Docker para desplegar Hornet rápidamente, aprovechando su portabilidad \cite{pullo_integrating_2024}.

Por su parte, si se considerase Ethereum, también existe la opción de nodos ligeros (les clientes) en dispositivos pequeños, pero en general Ethereum exige más recursos (un nodo completo de Ethereum es pesado para un Raspberry Pi sin medidas especiales). Sin embargo, para Ethereum se podría optar por usar APIs de terceros (Infura, Alchemy) en el prototipo, aunque eso sacrifica descentralización.

En Hyperledger Fabric, la RPi 4 puede correr algunos componentes (existen casos de usar Pis para pequeños peers en labs), pero para producción normalmente se emplean servidores más potentes.

En cuanto a IoT puro, es muy viable conectar sensores (temperatura en secadero, geolocalización de camiones, etc.) a una RPi actuando de gateway, que procese los datos y envíe transacciones a la DLT. Tecnologías como MQTT o HTTP podrían ser usadas localmente, y la RPi compone un mensaje estructurado (por ej., JSON-LD con evento EPCIS) y lo ancla en la red.

Una preocupación importante es la conectividad en entornos rurales: las dehesas tienen a veces mala conexión. Aquí, IOTA vuelve a tener un punto a favor: permite transacciones offline que luego se unen a la red (aunque requiere cierta sincronización), y en general tolera latencias. De hecho, IOTA ha explorado comunicaciones sobre LoRaWAN y otras redes IoT de baja potencia.

Otra pieza mencionada es el uso de testnet y herramientas Hornet para la arquitectura – en la fase de desarrollo, es recomendable probar en testnet para no arriesgar datos en mainnet y no incurrir en costos (en Ethereum esto es crucial porque en mainnet hay gas fees; en IOTA mainnet no hay fees pero conviene usar testnet por aislamiento). La viabilidad de nodos ligeros se extiende también al mantenimiento: Raspberry Pi es barato (unos 50€) pero hay que asegurar su estabilidad (las tarjetas SD tienden a degradarse por lo que quizás se podría usar un SSD externo, etc.).

A nivel software, tanto Hornet (IOTA) como posibles clientes ligeros Ethereum (Nethermind light, geth light) tienen documentación para ARM/RPi. En general, los avances en IoT y edge computing hacen posible desplegar <<DLT en el edge>>. Por ejemplo: existen nodos IOTA corriendo en \textit{gateways} IoT industriales que recogen datos de sensores y los escriben a la Tangle en tiempo real \cite{pullo_integrating_2024}. Incluso se puede concebir que en un secadero de jamones, el controlador climático (PLC) lleve embebido un cliente IOTA para registrar cada hora la temperatura/humedad en la ledger, como prueba de condiciones óptimas.

Si hablamos de otros DLTs: OriginTrail Node podría ser más pesado para RPi, pero han trabajado en que sea manejable; Hyperledger Fabric peer en RPi es posible pero con performance modesto. Ethereum 2.0 full node en RPi es difícil por espacio (varios cientos de GB), pero un node Ethereum light es posible. En este TFG, se priorizará IOTA, por la mención explícita de Hornet y Raspberry. Y efectivamente, esa elección es técnicamente sólida para un prototipo IoT + blockchain (además de accesible con hardware de bajo coste: IOTA fue concebida para IoT, proporciona SDKs en múltiples lenguajes, y la comunidad ha mostrado implementaciones exitosas en contextos agrícolas y logísticos \cite{pullo_integrating_2024}. Aprovechar testnets permite iterar sin riesgo, y luego se podría migrar a la mainnet una vez validado. Asimismo, herramientas como HORNET dashboard facilitan monitorear el nodo en la Pi vía web, lo que simplifica la operación técnica.

Un punto a planificar es la seguridad de los dispositivos IoT: una RPi conectada a internet debe endurecerse (usar VPN o firewall) para que el nodo DLT no sea comprometido. Pero estos son problemas abordables con buenas prácticas. Con todo, la propuesta de desplegar un nodo ligero IOTA (Hornet) en Raspberry Pi conectado a sensores IoT en un escenario de jamón ibérico es no solo factible sino que se alinea con casos de estudio recientes que muestran cómo la integración IoT+DLT mejora la sostenibilidad y la gestión agraria \cite{pullo_integrating_2024}.


\section{Gobierno de la Entidad y Privacidad: Definición de Roles (Consumidor, Auditor, Operador) y Cumplimiento RGPD}\label{sec:diseño-gobierno}