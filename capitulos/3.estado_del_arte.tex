\chapter{ESTADO DEL ARTE}\label{ch:estado}

\section{Tecnologías DLT e IoT Aplicadas a la Trazabilidad: Análisis Comparativo}\label{sec:estado-tecnologias}
Para implementar un Pasaporte Digital basado en blockchain o DLT en la cadena del jamón ibérico, es crucial elegir la plataforma adecuada. Diferentes tecnologías de Distributed Ledger ofrecen ventajas y compromisos en criterios como coste, rendimiento, sostenibilidad ambiental, complejidad técnica, interoperabilidad con IoT, etc. A continuación, se comparan cuatro candidatas relevantes:
\subsection{IOTA (Tangle)}\label{sec:estado-iota}
IOTA propone un Grafo Acíclico Dirigido (DAG) llamado Tangle en lugar de una blockchain tradicional \cite{pullo_integrating_2024-1}. Su principal ventaja es que las transacciones son \textit{feeless} (sin comisiones), lo cual es ideal para IoT donde puede haber multitud de microtransacciones de datos . Cada nuevo mensaje o transacción enviada ayuda a confirmar dos anteriores, eliminando la necesidad de mineros. Esto conlleva alta escalabilidad y muy baja huella energética, ya que no usa proof-of-work intensivo (IOTA actualmente funciona con un sistema coordinado y se encamina a ser totalmente descentralizado con \textit{Coordicide}). IOTA está diseñado para entornos de recursos limitados: por ejemplo, su nodo Hornet escrito en Go es muy ligero y puede correr en dispositivos modestos (Raspberry Pi, etc.) . De hecho, IOTA destaca por permitir potencialmente que los propios sensores o \textit{gateways} actúen como nodos ligeros, transfiriendo datos a la red casi en tiempo real. Otra ventaja es la sostenibilidad: al no haber minería, el consumo de energía por transacción es mínimo . En interoperabilidad, IOTA se complementa con estándares: soporta integraciones con EPCIS y otros formatos (OriginTrail, mencionada abajo, incluso permite usar IOTA como capa de anclaje). Como contra, IOTA es una tecnología relativamente joven y en evolución: requiere asegurar que la red (o una subred permisionada, si se usa en privado) tenga suficiente número de nodos para ser robusta. Sin embargo, para una aplicación piloto de un TFG, IOTA ofrece un entorno gratuito y abierto donde publicar eventos de trazabilidad sin incurrir en costes por cada uso, lo cual es muy atractivo para pymes.

\subsection{Ethereum}\label{sec:estado-ethereum}
Ethereum es la plataforma blockchain más popular para smart contracts. En un contexto de trazabilidad, permitiría crear tokens o NFT para cada jamón, o registros inmutables de cada etapa en contratos inteligentes. Su fortaleza es la seguridad y adoptabilidad: es muy conocida, tiene miles de desarrolladores y herramientas, y tras migrar a Proof of Stake (Ethereum 2.0) su consumo energético se ha reducido más de un 99\%, mejorando en sostenibilidad. No obstante, Ethereum pública tiene costes de transacción (gas) bastante altos y variables. Registrar muchos eventos (por ejemplo cada movimiento de miles de jamones)en la cadena principal sería económicamente inviable para una pyme, dado que cada transacción puede costar centavos o dólares según congestión. Esto se puede mitigar con soluciones de capa 2 (Polygon, etc.) o usando testnets, pero entonces se pierde algo de la seguridad de la red principal o se complica el diseño. En interoperabilidad, Ethereum soporta estándares de tokens (ERC-721, 1155) que algunas soluciones de cadena de suministro usan para representar productos. Sin embargo, Ethereum no está alineado de base con estándares sectoriales (habría que construir la capa de datos). A nivel complejidad, programar smart contracts seguros requiere expertise (riesgo de bugs costesos). Ethereum es potente y globalmente interoperable, pero su coste y rendimiento (15-30 TPS en mainnet), además de la complejidad de su ecosistema, la hacen menos práctica para el uso de sensores IoT masivo, salvo que se recurra a redes autorizadas o sidechains específicas \cite{noauthor_que_nodate}.

\subsection{Hyperledger Fabric}\label{sec:estado-fabric}
Hyperledger Fabric es una plataforma DLT privada/consorcial, muy utilizada en entornos empresariales (p.ej. la red IBM Food Trust para trazabilidad alimentaria está sobre Fabric). Sus ventajas incluyen control de participantes (solo entidades autorizadas pueden ejecutar nodos y transacciones), alto rendimiento personalizable (decenas a cientos de TPS, sin fees porque no hay criptomoneda nativa) y flexibilidad modular (se pueden definir canales privados, distintos algoritmos de consenso, etc.). Para una cadena de suministro de jamón ibérico, una red Hyperledger podría involucrar a ganaderos, secaderos, distribuidores y reguladores, cada uno con un nodo, garantizando que solo ellos ven ciertos datos sensibles pero compartiendo la trazabilidad de forma confiable. Al ser permisionada, es más fácil cumplir requisitos de confidencialidad que en una pública. Además, Fabric permite integrar identidades corporativas existentes (PKI) y es tolerante a \textit{pluggable consensus} (por defecto usa PBFT). Como contrapartida, Fabric requiere una infraestructura y gobierno consorcial, es decir, habrá que convencer a todos los actores de montar y mantener nodos, definir quién administra la red, etc. Esto puede ser complicado si no hay un actor dominante que lidere (en Food Trust, IBM actúa como proveedor central). También carece de la transparencia pública de las blockchains públicas, lo que podría restar confianza al consumidor final si no se le da acceso a la información (aunque se puede hacer públicas ciertas pruebas hash). En cuanto a IoT, Fabric puede recibir datos IoT pero generalmente a través de APIs centralizadas que escriben en la cadena; no está pensado para ejecutar directamente en dispositivos pequeños. En coseso, Fabric no tiene fees por transacción, pero su CAPEX/OPEX recae en gestionar la red (servidores, certificación, desarrollo a medida). Resumiendo, Hyperledger Fabric es ideal si se quiere control y rendimiento en un entorno cerrado (por ejemplo, un piloto entre dos o tres empresas grandes), pero para un enfoque abierto e integrador de muchos productores pequeños quizás sea menos adecuada sin un ente coordinador \cite{noauthor_blockchain_nodate-2}.

\subsection{OriginTrail}\label{sec:estado-origin}
OriginTrail es algo diferente: es un protocolo descentralizado enfocado específicamente en datos de la cadena de suministro. Funciona como una Red de Grafos Descentralizados (DKG) donde la información de trazabilidad se almacena de forma distribuida y se valida en una blockchain subyacente. Inicialmente se utilizó Ethereum para anclar hashes de datos EPCIS y asegurar la inmutabilidad, pero ahora opera su propia blockchain basada en Polkadot. Su gran ventaja es que nativamente soporta estándares GS1: por ejemplo, estructuras de datos EPCIS 2.0 y GS1 CBV (Core Business Vocabulary) están incorporadas en el modelo de datos de OriginTrail \cite{noauthor_whats_2025}\cite{rakic_first_nodate}. Esto significa que es muy fácil para empresas que ya usan GS1 (como muchas alimentarias) publicar sus eventos de trazabilidad en OriginTrail sin transformaciones complejas. Además, OriginTrail permite privacidad selectiva: los datos detallados pueden permanecer off-chain en nodos descentralizados, compartidos solo con quien corresponda, mientras que solo sus huellas (hashes) se anclan en la blockchain para asegurar integridad \cite{noauthor_origintrail_nodate}. En términos de coste, OriginTrail utiliza un mecanismo de staking y mercado de datos: las empresas pagarían a nodos por alojar/servir sus datos, normalmente usando el token TRAC, pero los costes han sido hasta ahora relativamente bajos comparados con las tasas de gas de Ethereum porque la mayoría del almacenamiento es off-chain. Para IoT, OriginTrail podría actuar más como el \textit{middleware}: los sensores envían datos a un sistema IoT (una base de datos en la nube), y luego esos registros se suben al DKG de OriginTrail donde son inmutabilizados. La complejidad de OriginTrail es moderada, pero requiere operar un nodo de la red DKG o confiar en un proveedor que lo haga. Otro punto muy favorable es que ya ha sido ya probado en casos reales: el operador ferroviario suizo SBB usa OriginTrail+EPCIS para rastrear mantenimiento de activos , y marcas de ropa lo han usado para trazabilidad textil. Para nuestro caso, OriginTrail ofrece interoperabilidad lista para usar con GS1 Digital Link, EPCIS y Verifiable Credentials , lo que encaja perfectamente con los objetivos del Pasaporte Digital. La desventaja podría ser que es una tecnología menos conocida que Ethereum o Fabric, y depende de una red cuyos nodos están principalmente fuera de España, lo que podría preocupar a algunos (pero al mismo tiempo es descentralizada globalmente).

Comparando las DLTs presentadas, IOTA destaca por coste cero por transacción y enfoque IoT; Ethereum por su ecosistema maduro pero con coste por transacción y escalabilidad limitada (a menos que se use \textit{layer 2}); Hyperledger Fabric por control privado y rapidez pero menos transparente; OriginTrail por especialización en supply chain con estándares integrados. Para un proyecto de jamón ibérico, si se prioriza involucrar a muchos pequeños productores y sensores IoT, IOTA u OriginTrail parecen opciones más alineadas (feeless e integrables con GS1 respectivamente). Si se prioriza robustez y confianza de marca, quizá una red consorcial (Fabric) entre grandes jamoneros y ASICI podría funcionar, pero perdería apertura. Una estrategia posible es combinar OriginTrail DKG para manejar los datos estandarizados de trazabilidad, y anclar hashes tanto en IOTA (pública, barata) como en una red permisionada (para duplicar confianza). De hecho, la complementariedad es tendencia: IOTA se ve como complemento de blockchains para IoT, no necesariamente reemplazo \cite{pullo_integrating_2024}.