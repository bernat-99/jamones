\chapter{INTRODUCCIÓN}\label{ch:intro}

\section{Antecedentes y Justificación del DPP para Jamón Ibérico}\label{sec:intro-antecedentes}

\subsection{Trazabilidad Alimentaria: Sistemas Tradicionales, certificación y Problemas de Fraude}\label{sec:intro-trazabilidad}
En el sector ibérico ya existen mecanismos de identificación y registro que cumplen con la trazabilidad legal mínima: lotes, albaranes, precintos vinculados a la Norma de Calidad y, en su caso, controles adicionales de las DOP/IGP. Sistemas sectoriales como ÍTACA agregan parte de esta información y facilitan su consulta por operadores y autoridades \cite{}. Sin embargo, la evidencia apunta a limitaciones recurrentes: fragmentación de fuentes (\textit{silos} de datos en ERP y hojas de cálculo), correlación manual entre documentos, verificación \textit{ex post} ante incidencias, asimetría informativa para el consumidor y escasa granularidad legible por máquina para auditorías digitales \cite{}.

El DPP propuesto no sustituye estos registros, sino que los ordena y los hace verificables: define un identificador global resoluble (GS1 Digital Link), normaliza eventos de trazabilidad (EPCIS 2.0), ancla resúmenes criptográficos en una DLT (Tecnología de Registro Distribuido, por sus siglas en inglés) sin comisiones (IOTA Tangle) y publica vistas diferenciadas por rol. Así, se reduce el tiempo de investigación de expedientes, se acota el riesgo de error humano en la inserción documental y se incrementa la confianza del mercado en la autenticidad del jamón ibérico \cite{}.

\subsection{Competitividad}\label{sec:intro-competitividad}
El jamón ibérico es un emblema gastronómico español y su creciente fama global trae oportunidades, pero también retos competitivos. Uno de los riesgos recientes es la aparición de jamones \textit{ibéricos} producidos en otros países, especialmente China:

Desde hace más de una década se conoce el interés chino por replicar el jamón español: ya en 2010-2011, empresas chinas comenzaron a criar cerdos de razas similares al ibérico e implementar técnicas de salado y curado aprendidas de productores españoles \cite{noauthor_jamon_2011}. De hecho, productores españoles sospechan que delegaciones chinas que visitaron sus fábricas hace años (oficialmente para habilitar importaciones) recopilaron know-how que luego aplicaron en plantas de jamón en China . El resultado han sido los primeros \textit{jamones ibéricos made in China}\cite{noauthor_jamon_2011}.

Si bien su calidad organoléptica aún dista de la de un auténtico ibérico español, a la vista pueden parecerse mucho, al menos a ojos de un consumidor poco experto. Esto supone un riesgo de plagio comercial: jamones chinos de menor calidad y precio inundando mercados asiáticos, haciendo sombra al producto español auténtico \cite{noauthor_jamon_2011}. Ya en 2018 se reportaba que jamones chinos estaban logrando un sabor aproximado al serrano estándar, con tiempos de curación menores, y los productores españoles manifestaban preocupación por ser desplazados en Asia por ese producto más barato \cite{noauthor_china_nodate}. Además, existe la posibilidad de uso fraudulento de términos protegidos: por ejemplo, llamar \textit{jamón ibérico} en China a un jamón de raza local, aprovechando la falta de regulación en el país. Aunque la denominación \textit{Jamón Ibérico} está protegida en la UE (y en China, en principio, solo puede usarse para importados legítimos a través de acuerdos bilaterales), en la práctica la protección legal internacional es limitada.

Este contexto hace urgente proteger la autenticidad: mecanismos como la trazabilidad avanzada, DLT y DPP pueden garantizar la identidad de origen de cada pieza, de modo que en cualquier parte del mundo se pueda verificar si un jamón con etiqueta \textit{Ibérico de España} realmente proviene de una granja y secadero españoles certificados.

Otra arista competitiva es la expansión de la demanda internacional: China es también un gran comprador de jamón español (tras la apertura de exportaciones, se ha vuelto un mercado importante con crecimiento de  más de un 100\% en cinco años) \cite{depares_china_2024}\cite{bonet_preguntar_2023}. Esto ha motivado a ASICI y la UE a invertir en promoción exterior, por ejemplo la campaña \textit{Jamones Ibéricos de España, Embajadores de Europa} con giras en varias ciudades chinas \cite{depares_china_2024}. El objetivo es afianzar la imagen de prestigio del ibérico y educar al consumidor chino para que reconozca el auténtico. El Pasaporte Digital podría ser una herramienta en estas campañas: incitar a un comprador extranjero que escanee el QR del jamón y compruebe datos de trazabilidad, certificaciones, incluso un saludo del maestro jamonero en la dehesa (algo difícil de falsificar por imitadores). Asimismo, un DPP basado en blockchain aportaría inamovibilidad de datos: un competidor no podría simular fácilmente la información de años de crianza y curación que están registradas en una cadena de bloques pública.

Más allá de China, hay otros competidores: por ejemplo, en Italia algunos productores han empezado a criar cerdos de raza ibérica importados, intentando producir \textit{jamón ibérico italiano}. Y países como USA han desarrollado sus propios jamones curados tipo \textit{serrano} (aunque no ibéricos aún).

España debe por tanto proteger la \textit{Marca Ibérico} como sinónimo de calidad y origen único, similar a cómo la Champaña francesa se protege. Las denominaciones de origen (ej. DOP Jabugo) ayudan, pero no cubren a todo ibérico (que también se produce fuera de DOP). Por eso, la marca colectiva \textit{Ibérico de España} y herramientas tecnológicas de trazabilidad son clave. En 2024 hubo casos de intervención en frontera de productos cárnicos etiquetados engañosamente: se reportó la incautación en Mallorca de un lote importado de China con etiquetado irregular que simulaba jamón español \cite{depares_china_2024}. Esto demuestra que el riesgo es real y actual.

Todo lo anterior demuestra que la competencia internacional (legal o ilícita) exige redoblar esfuerzos en trazabilidad, autenticación y diferenciación. Un Pasaporte Digital de Producto fortalecería la posición competitiva del jamón ibérico español al proporcionar transparencia total y verificación instantánea de autenticidad, dificultando las falsificaciones y aportando un argumento de valor añadido en mercados globales. Además, permitiría destacar las cualidades únicas (dehesa, bellota, tradición) frente a imitaciones más industriales, protegiendo así la reputación ganada por la \textit{Marca España} en este producto gourmet.

\subsection{Casos de Uso y Comparativas Relevantes de Pasaportes Digitales}\label{sec:intro-antecedentes}
Aunque el concepto de Digital Product Passport es relativamente nuevo, ya existen diversas iniciativas y casos de uso en distintos sectores que sirven de referencia para el caso del jamón ibérico. A continuación, se identifican algunos ejemplos reales de implementación de pasaportes digitales (en sectores preferentemente distintos al cárnico), las lecciones aprendidas y cómo podrían adaptarse al contexto ibérico:
\begin{itemize}
	\item Pasaporte Digital en el sector textil (moda sostenible): La industria de la moda ha sido pionera en probar los DPP para proveer transparencia sobre origen de materiales y condiciones de producción. Un ejemplo es la marca británica Nobody’s Child, que ha implementado pasaportes digitales en sus prendas mediante un código QR en la etiqueta de cuidado. Al escanearlo, el cliente puede explorar la cadena de suministro de la prenda: desde la procedencia de la tela, fábrica de confección, hasta datos de impacto ambiental (agua, CO\textsubscript{2)} de esa prenda. Este proyecto, en colaboración con la plataforma Fabacus, busca no solo cumplir con futuras regulaciones sino \textit{conectar cómo se hizo el producto con el cliente}. La lección aquí es la importancia de la conectividad: el QR lleva a una experiencia interactiva y fácil de entender para el consumidor, convirtiendo la trazabilidad en una historia de marca. De igual forma se podría aplicar al jamón ibérico, un QR en la vitola del jamón podría llevar a una página o \textit{pass app} con un recorrido \textit{De la dehesa a tu mesa}, mostrando la granja, la bodega de curación, y métricas de sostenibilidad. Este caso demuestra que los consumidores sí escanearán un código si ofrece información interesante, y refuerza la confianza en la marca (Nobody’s Child se posiciona así como transparente y responsable). Además, a nivel técnico, está construido sobre GS1 Digital Link y JSON-LD, similar a lo que necesitaría un DPP alimentario, lo cual valida la elección de esos estándares \cite{noauthor_10_nodate}.
    \item Impact Receipts y transparencia de huella (ASKET, moda): La marca sueca ASKET introdujo en 2023 la “Impact Receipt”, un recibo de compra detallando la huella ambiental de cada prenda (CO\textsubscript{2}, agua, energía). Esto acompaña al producto vendiéndose, mostrando al cliente el “precio oculto” en recursos naturales. Si bien no es un pasaporte en sí mismo, va en la línea de lo que los DPP deben hacer: comunicar indicadores de sostenibilidad. En la implementación de ASKET, integran datos de todo el ciclo productivo. Lo que queda demostrado con este ejemplo es que presentar datos de impacto de forma comprensible añade valor educativo y posiblemente cambia el comportamiento del consumidor hacia más responsabilidad. Para el jamón ibérico, un DPP podría incluir algo semejante a una “etiqueta de impacto” alimentario: e.g. “Este jamón implicó 30 kg CO\textsubscript{2}e de emisiones, 1500 L de agua, y apoya un ecosistema de dehesa que almacena CO\textsubscript{2}”. Traducir la sostenibilidad a números concretos hace más tangible el concepto al comprador. La experiencia de ASKET sugiere que la transparencia radical puede ser bien recibida y generar diálogo (sus fundadores dicen que es para combatir la sobreconsumo haciendo visible el coste real) \cite{noauthor_10_nodate}.
    \item Digital ID para circularidad en moda (PANGAIA ReWear): La empresa PANGAIA, conocida por innovación sostenible, implementó pasaportes digitales en sus prendas con tecnología EON (digital ID) y lanzó PANGAIA ReWear, una plataforma de reventa de segunda mano habilitada por esos pasaportes. Cada prenda tiene un QR único; al escanearlo, se accede al historial del producto. Además, si un cliente quiere revender la prenda, la plataforma toma automáticamente la información del pasaporte (materiales, imágenes, etc.) para listar el producto, facilitando la reventa. Esto es muy innovador porque cierra el ciclo: el DPP no solo informa al primer comprador, sino que viaja con el producto a su siguiente dueño, extendiendo su vida útil. Un pasaporte digital puede habilitar modelos de negocio circulares (reventa, alquiler, reciclaje) al eliminar fricciones de información. Si bien un alimento no se revende para consumo, esta idea se podría trasladar a, por ejemplo, reutilización de envases o certificados de calidad reutilizables. Un caso de uso más cercano es el que hace PANGAIA del pasaporte para proveer consejos de cuidado y reciclaje de la prenda, análogo a cómo un DPP de jamón puede dar consejos de conservación (cómo guardar el jamón, recetas para aprovechar restos, etc.) y opciones de disposición final (reciclar envoltorio). PANGAIA demuestra que los pasaportes pueden crear una comunidad alrededor del producto (vendedores y compradores de segunda mano confiando en la info del DPP). En alimentos, esto podría transformarse en confianza en intercambios B2B (un distribuidor extranjero confía en la información del DPP sin necesidad de certificaciones físicas adicionales) \cite{noauthor_10_nodate}.
    \item Pasaporte de baterías (sector automoción, Global Battery Alliance): Uno de los casos más directamente relacionados con la idea de DPP en la UE es el Battery Passport. La Global Battery Alliance (GBA), que incluye empresas como Ford, Tesla, etc., ha pasados pruebas piloto de un pasaporte digital para baterías de vehículos eléctricos. Usan blockchain, IA y auto-ID para documentar toda la vida de una batería: desde la extracción de minerales (cobalto, litio) hasta la fabricación, uso en coche, y reciclaje final. Ford, por ejemplo, colabora con Everledger en un piloto de este tipo, buscando asegurar origen responsable de materiales y trazabilidad hasta reciclaje \cite{noauthor_10_nodate}. Esta iniciativa está muy alineada con regulaciones: la UE aprobó en 2023 el Reglamento de baterías, que exige un pasaporte digital para baterías de EV a partir de 2026 \cite{noauthor_boees_nodate-3}. Las enseñanzas que nos da este caso son:
    \begin{enumerate}
	\item Gran complejidad de la cadena: integrar datos de múltiples proveedores, cada uno con su sistema, requirió definir un estándar común de datos (en GBA desarrollaron un Data Blueprint). El equivalente en el jamón sería integrar datos de ganaderos, mataderos, secaderos; es una cadena más corta que la de una batería, pero similar desafío de interoperabilidad.
    \item Importancia de la verificación de sostenibilidad: El battery passport se enfoca en confirmar que ciertos criterios se cumplieron (materiales libres de trabajo infantil, contenido de reciclado, etc.). Para el jamón, un DPP podría incluir verificación de que el cerdo pastó X meses o que se respetó bienestar animal – quizás auditado por terceros y subido al DPP como certificado.
    \item Uso de blockchain para confianza multi-actor: En la batería, competidores colaboran pero no confían ciegamente, por eso blockchain es útil para crear registro compartido. En la cadena ibérica, ya hay más confianza central, pero si se integran otros (distribuidores internacionales, autoridades de distintos países), tener un ledger neutro puede ser valioso. El jamón ibérico podría inspirarse en GBA en cuanto a gobernanza: la interprofesional ASICI podría hacer de “alianza” para desarrollar el pasaporte de jamón, similar al rol de GBA en baterías, reuniendo a todos los stakeholders (productores, certificadoras, tecnología). Esto facilitaría la adopción sectorial \cite{noauthor_10_nodate}.
\end{enumerate}
    \item Grandes minoristas y trazabilidad digital (Decathlon, alimentos varias): Fuera del DPP formal, es relevante mencionar cómo los gigantes del retail abordan la trazabilidad, porque impulsan a los proveedores. Decathlon, por ejemplo, implementó RFID en 50.000 puntos (fábricas, almacenes, tiendas) para rastrear todos sus productos deportivos  Si bien su motivación es más logística, esa infraestructura permite que un cliente escanee un tag RFID/QR y obtenga info del producto \cite{noauthor_decathlon_nodate}. En alimentación, Walmart (EEUU) exigió a sus proveedores de hojas verdes unirse a su plataforma IBM Food Trust (blockchain) para trazabilidad después de brotes de E.coli \cite{noauthor_how_nodate}. El resultado fue que algunos proveedores como Zespri (kiwis) usan blockchain para trazabilidad integral, reduciendo tiempos de respuesta en retiradas de días a segundos. Cuando un líder de mercado impone un estándar de trazabilidad digital, los productores deben adaptarse o arriesgan perder ese canal. Para el jamón ibérico, imaginemos que en unos años, cadenas como Carrefour o El Corte Inglés dijeran “solo compraremos ibéricos con pasaporte digital conforme a la normativa UE”. Estar preparados con un sistema interoperable (idealmente basado en GS1/EU standards) sería crucial para mantener mercados. Además, casos como IBM Food Trust mostraron que la trazabilidad blockchain mejoró la eficiencia: Walmart logró rastrear un mango del supermercado a la finca en 2.2 segundos usando blockchain, cuando antes les tomaba 7 días con métodos tradicionales \cite{noauthor_how_nodate}. Aunque el jamón no es producto de retiradas urgentes (no es perecedero inmediato), esta eficiencia en recall es valiosa para seguridad alimentaria.
    \item Ejemplos en alimentación premium: Dentro del sector alimentario, hay algunos ejemplos más cercanos (empresas de vinos y espirituosos han experimentado con DPP o similares). En el sector vitivinícola, la startup Wine Blockchain en Italia permitió a consumidores escanear una botella y ver desde el viñedo hasta la bodega. Incluso la UE ya exige que a partir de diciembre 2023 las etiquetas de vino tengan información digital accesible (ingredientes, calorías) mediante QR \cite{motusic_understanding_2024}. Ese es un precursor de DPP en bebidas. La digitalización de la información de producto puede ser impulsada mediante regulación (como en vino) y normalmente comienza por datos que antes no se daban (nutricionales, en este caso). En el jamón ibérico, podría ocurrir algo parecido, quizás la normativa futura exija dar al consumidor información sobre huella ambiental o bienestar; un DPP facilita esa provisión sin sobrecargar la etiqueta física. Otro ejemplo es el de productores de aceite de oliva premium empleando QR blockchain para certificar origen de las aceitunas y fecha de molturación, creando confianza en mercados asiáticos donde hay falsificaciones. Esto es similar a la necesidad de jamón en China: se han detectado falsos \textit{5J Joselito} \cite{guardia_civil_incautadas_2023}, etc. Y algunas marcas han puesto sistemas de autenticación digitales (códigos alfanuméricos únicos bajo rascado). Un DPP más completo reemplazaría esos sistemas aislados con uno estándar.
\end{itemize}

De los usos enumerados, se pueden extraer lecciones globales y  mejores prácticas aplicables al jamón ibérico:
\begin{enumerate}
	\item Centrarse en la historia y la conexión emocional. No saturar al usuario final con datos técnicos, sino presentar la información del pasaporte de forma narrativa y visual (como la moda lo hace mostrando el recorrido de la prenda, o un vídeo del chef embajador hablando del jamón en la campaña en Hong Kong) \cite{depares_china_2024}. En el DPP del jamón, podría haber pequeños vídeos o fotos de la dehesa al atardecer, del maestro jamonero realizando el calado, etc., para darle vida a los datos.
    \item Interactividad y utilidad: Hacer que el pasaporte no sea solo un PDF estático, sino una plataforma interactiva. Por ejemplo, permitir al consumidor hacer preguntas al chatbot sobre el producto, o registrarse para recibir un certificado de autenticidad a su nombre, etc; esto aumenta el \textit{engagement}. Pangaia mostró que la utilidad (facilitar reventa) fomenta el uso del pasaporte \cite{noauthor_10_nodate-1}. En el jamón, donde no existe casi posibilidad de reventa (se contempla la posibilida de sorteos y trofeos en campeonatos), podría haber un esquema de fidelización: escaneas el DPP y acumulas puntos si reciclas el envase o si completas una encuesta de cata, etc.
    \item Estandarización y colaboració.: Todos los casos de éxito han surgido de colaboraciones: GBA para baterías con un estándar de datos común, proyectos con GS1 en moda, etc. Para jamón ibérico, la colaboración entre productores, la interprofesional, organismos certificadores y \textit{tech partners} es crucial. Esto asegurará que el DPP propuesto cumpla normativas UE (ESPR, etc.) y sea reconocido. Por ejemplo, en el sector textil la Comisión Europea encargó un estudio (2022) sobre DPP en textiles para guiar la implementación \cite{european_parliament_directorate_general_for_parliamentary_research_services_digital_2024-1}; sería conveniente que el sector cárnico se anticipe creando sus propios criterios antes de que se los impongan.
    \item Privacidad de datos empresariales. Un aprendizaje de proyectos en otros sectores es equilibrar transparencia con resguardo de secretos comerciales. No todo dato interno debe hacerse público en el DPP. Por ejemplo, Ford no va a publicar la receta exacta de su batería, pero sí los indicadores exigidos (contenido reciclado, huella CO\textsubscript{2}) \cite{noauthor_10_nodate-1}. Similarmente, un productor de jamón podría no querer revelar la ubicación exacta de todas sus fincas (por seguridad o competencia) – en su lugar, podría indicar región y características sin dar coordenadas precisas; igualmente, puede no querer decir sus proveedores de pienso. El DPP debe estructurarse de modo que cumpla con la normativa (que pedirá ciertos datos obligatorios) pero proteja información sensible. Se pueden usar \textit{verifiable credentials} o certificaciones de terceros en el pasaporte en lugar de datos crudos: por ejemplo, en vez de publicar la lista de campos donde pastaron, incluir un sello digital de autoridad (Junta Andalucía) que certifica “este jamón proviene de una dehesa inscrita y supervisada”. Esto da confianza sin exponer todo detalle.
    \item Afrontar la brecha digital del consumidor. No todos los consumidores de jamón (sobre todo generaciones mayores) van a escanear códigos QR o entender un pasaporte digital complejo. Hay que diseñar con simplicidad. El pasaporte debería ofrecer valor incluso a quien no es experto: por ejemplo, un simple “semáforo” de sostenibilidad, o una breve historia en texto más multimedia, y luego un botón “ver datos técnicos” para quien sí quiere profundizar. La experiencia de lecherías y otros es que demasiada información puede abrumar, así que la interfaz debe ser clara (el marco GS1 Digital Link sugiere esto, con capa para máquinas y otra presentable a humanos).
    \item Resultados medibles. De los pilotos existentes, conviene observar resultados: ¿aumentaron ventas? ¿mejoró la confianza? Algunos son muy nuevos para saberlo. No obstante, hay indicios: H\&M con su Digital ID dijo que buscaba ofrecer servicios postventa (reciclaje, etc.) para fidelizar clientes \cite{noauthor_10_nodate-1}. Si el jamón ibérico implementa un DPP, se podría medir el efecto en protección de marca en mercados lejanos (menos reclamaciones de falsificación, etc.), o en eficiencia (tiempo de auditorías reducido porque todo está en sistema). Documentar estos beneficios ayudará a la expansión del proyecto dentro del sector.
\end{enumerate}

En conclusión, los casos de uso reales muestran que el Pasaporte Digital de Producto no es ciencia ficción, sino una tendencia en marcha en diversas industrias. En sectores como moda, automoción, electrónica y alimentación de lujo ya se está comprobando su valor para transparencia, cumplimiento regulatorio y habilitación de modelos circulares. El sector del jamón ibérico, con su fuerte componente de calidad diferenciada y necesidad de proteger autenticidad, puede aprovechar esas lecciones: implementando un DPP que combine información rigurosa (validada por blockchain) con una narrativa atractiva, asegurando interoperabilidad con estándares UE, y anticipando los requerimientos futuros de sostenibilidad y trazabilidad que vienen en el marco normativo \cite{noauthor_digital_nodate-1}. De esta forma, el jamón ibérico podría posicionarse internacionalmente no solo como un producto tradicional exquisito, sino también como un producto trazable 100\% transparente y sostenible, alineado con las exigencias del consumidor y la sociedad del siglo XXI. Cada jamón llevaría no solo su aroma y sabor, sino también su “historia digital” imborrable, fortaleciendo la confianza y el valor de la marca ibérica en cualquier lugar del mundo.

\section{Diferencias entre Jamón Ibérico y Jamón Serrano y Confusiones de Mercado}\label{sec:intro-diferencias}

\subsection{Diferencias técnico-legales}
Aunque ambos son jamones curados tradicionales de España, el jamón ibérico y el jamón serrano se diferencian por la raza del cerdo, su alimentación, normativas de calidad y denominaciones asociadas.

El jamón ibérico proviene exclusivamente de cerdos de raza ibérica (negros) o cruzados con al menos 50\% de sangre ibérica, criados bajo las condiciones definidas por la Norma de Calidad del Ibérico (Real Decreto 4/2014) \cite{ministerio_de_agricultura_alimentacion_y_medio_ambiente_real_2014}. Esta norma española establece categorías según la pureza racial (100\% ibérico, 75\% o 50\%) y el tipo de alimentación del cerdo (bellota, cebo de campo, cebo) \cite{noauthor_precintos_nodate}. Cada combinación se identifica con un precinto de color en la pata del jamón (negro, rojo, verde o blanco) que garantiza y certifica ante el consumidor la categoría del producto \cite{noauthor_precintos_nodate}. Además, la trazabilidad del ibérico está muy vigilada: los cerdos deben nacer, criarse y sacrificarse en explotaciones registradas; se controla su alimentación (e.g. tiempo mínimo en montanera) y el tiempo de curación mínimo (por ejemplo, un jamón ibérico de bellota suele curar más de 36 meses) \cite{ministerio_de_agricultura_alimentacion_y_medio_ambiente_real_2014}.

Por su parte, el jamón serrano típicamente proviene de cerdos de raza blanca (Landrace, Large White, Duroc) y sigue un proceso de curado más corto (mínimo de 7 a 12 meses según peso). Existe una especialidad tradicional reconocida por la UE llamada ETG Jamón Serrano (Especialidad Tradicional Garantizada) que estipula ciertos estándares de elaboración para usar la denominación \textit{Serrano} \cite{noauthor_etg_nodate}. Sin embargo, \textit{jamón serrano} en general no es una DOP, sino un término genérico para jamón curado de cerdo blanco, por lo que su regulación de base es la normativa general de alimentos y carnes.

Las diferencias técnicas incluyen que el serrano suele tener menor infiltración grasa y un sabor más suave, dado que la raza y la alimentación (piensos) difieren de los ibéricos de bellota que acumulan ácido oleico en su grasa por la dieta de bellotas.

Legalmente, la Norma de Calidad del Ibérico no aplica a jamones \textit{serranos}; estos pueden acoger otras figuras de calidad: por ejemplo, Jamón de Teruel es una DOP de jamón serrano de Aragón con pliego propio (cerdos raza Duroc x Landrace, criados y curados en Teruel) \cite{ministerio_de_agricultura_pesca_y_alimentacion_espana_pliego_2022-1}. Entonces, mientras un jamón serrano estándar no tiene un sistema de precintos obligatorio, el jamón ibérico sí lo tiene – es obligatorio colocar el precinto de norma a cada pieza ibérica en el matadero, con un código único trazable en la base de datos ÍTACA \cite{noauthor_sistema_nodate}.

En términos técnico-legales: existen diferencias de raza y cría (ibérico y blanco), regulación específica nacional para ibérico y especialidad tradicional para serrano, sistemas de trazabilidad y certificación distintos, y protección del término \textit{ibérico} dentro de España. Esto se traduce en el mercado a productos de segmento distinto: el jamón ibérico es un producto de calidad de mayor precio, con producción limitada y ligada a territorios concretos por sus DOPs e IGPs (Extremadura, Andalucía, Salamanca etc.), mientras que el \textit{serrano} se produce a gran escala en toda España (y otros países llegan a producir jamón de cerdo blanco similar, como \textit{prosciutto} en Italia o jamones \textit{tipo serrano} en USA o incluso China).

\subsection{Confusión terminológica en etiquetado, percepción y mercado}
A pesar de las normas, persisten confusiones y usos indebidos de términos en la comercialización de jamones, lo que afecta la percepción del consumidor. Un caso clásico es el término \textit{pata negra}, popularmente asociado al mejor jamón ibérico. Desde 2014, legalmente \textit{Pata Negra} está reservado únicamente para jamón de bellota 100\% ibérico (precinto negro) en España. Sin embargo, en el lenguaje coloquial e incluso en publicidad internacional, a veces se llama \textit{pata negra} a cualquier jamón español de calidad, lo cual puede inducir a error. Otra confusión surge entre la etiqueta y el precinto: muchos consumidores no saben que el color del precinto (cinta de plástico en la caña del jamón) tiene un significado oficial; a veces se fijan solo en la marca comercial o en expresiones como \textit{reserva} o \textit{gran selección} que pueden figurar en la etiqueta del fabricante sin un estándar claro. Por ello, las autoridades y ASICI han hecho campañas informativas enfatizando: \textit{Fíjate en el color del precinto, no te dejes llevar solo por la etiqueta}\cite{noauthor_consumo_2022}. Cada precinto lleva además un código de barras único ligado al registro ÍTACA, que puede escanearse con la “APP Ibérico” de ASICI para verificar la trazabilidad de esa pieza. En la siguiente imagen, se ilustran los cuatro precintos de norma con sus colores y categorías correspondientes (negro, rojo, verde, blanco), que son la única forma fiable de identificar la calidad o categoría de un jamón ibérico en el punto de venta \cite{noauthor_precintos_nodate}:

Figura: Precintos oficiales del Jamón Ibérico según Norma de Calidad (negro: bellota 100\% ibérico; rojo: bellota ibérico 75\% o 50\%; verde: cebo de campo ibérico; blanco: cebo ibérico).

La percepción de los consumidores a veces mezcla \textit{ibérico} y \textit{serrano} por simple desconocimiento. En mercados extranjeros, el término \textit{jamón ibérico} ha ganado fama y en ocasiones comerciantes inescrupulosos podrían denominar \textit{ibérico} a jamones que no lo son. Para atajar esto, España protege la marca \textit{Jamón España} y las denominaciones de origen ibéricas (DOP Guijuelo, DOP Jabugo, etc.) en tratados internacionales, pero aun así es un desafío educar al mercado.

Otra fuente de confusión es que los jamones de Denominaciones de Origen Protegidas (DOP) ibéricas (como Jabugo, Guijuelo) llevan sus propios precintos y vitolas de consejo regulador: aunque respetan el mismo código de colores de la Norma, el consumidor podría no entender que un precinto blanco de DOP Dehesa de Extremadura también indica cebo ibérico. ASICI recomienda que el comprador \textit{Siempre busque el logotipo de ASICI en el precinto de norma de calidad} cuando adquiera un ibérico estánda (las DOP llevan su sello particular) \cite{noauthor_precintos_nodate}\cite{ministerio_de_agricultura_alimentacion_y_medio_ambiente_real_2014}.

Queda claro que existe cierta terminología compleja (porcentajes raciales, nombres de alimentación) que puede resultar confusa sin información accesible. Aquí es donde un Pasaporte Digital podría aportar claridad: escaneando un código, el consumidor podría ver en su idioma una explicación sencilla del producto: e.g. “Jamón Ibérico de Cebo – Cerdo 50\% raza ibérica alimentado con pienso en granja. Curación: 24 meses en Badajoz. Granja certificada en bienestar animal.”, eliminando ambigüedades. Esto mejoraría la transparencia y la confianza, combatiendo la confusión intencionada o accidental en el etiquetado.

También hay un aspecto de percepción de calidad: el jamón serrano de alta gama (por ejemplo, de Teruel) puede ser excelente, pero muchos consumidores asumen que ibérico es siempre superior. En realidad, son productos diferentes, y cada uno con gradaciones internas. Educar al público en esas diferencias es parte de la labor del sector. Un hallazgo de la Secretaría de Consumo en España fue que una proporción significativa de consumidores no distingue la raza (100\% vs 50\% ibérico) con solo ver la etiqueta, de ahí la importancia del sistema de precintos de colores introducido en 2014 \cite{noauthor_consumo_2022}.

Por último, en mercados internacionales pueden darse traducciones o naming confusos: en China al jamón curado español se le llama a veces \textit{jamón Parma español} o similares, lo que mezcla denominaciones. La creación de una identidad digital unívoca por producto (pasaporte) ayudaría a uniformar la información presentada en todos los mercados, evitando que en cada país se use una terminología distinta que pueda crear confusión.

\section{Cadena de Suministro y Cadena de Valor del Jamón Ibérico y su Contexto Comercial}\label{sec:intro-cadena}
\subsection{Etapas, Sensórica y Trazabilidad}
El proceso productivo del jamón ibérico abarca una secuencia de etapas bien definidas, cada una de las cuales añade valor y requiere controles de calidad y trazabilidad específicos. A continuación, se describen las principales fases de la cadena de valor del jamón ibérico, identificando sus particularidades y necesidades de sensorización/trazabilidad:
\begin{enumerate}
	\item Genética y Cría Inicial: todo comienza con la selección genética de los reproductores. En el ibérico, asegurar la pureza racial (100\%, 75\% o 50\% ibérico según el producto final que se quiere obtener) es crucial. Las granjas reproductoras mantienen libros genealógicos y hoy día usan chips o crotales RFID en las madres con el propósito de mantener la identificación. Una cerda ibérica pare generalmente en primavera; momento en el que los lechones son identificados individualmente. Aquí la trazabilidad se inicia registrando el origen de cada animal (pedigrí, explotación de nacimiento). Sensórica en esta etapa no tiene mucho sentido. Los datos genéticos y de nacimiento se registran en ÍTACA y podrían integrarse al DPP (por ejemplo, \% de raza ibérica del cerdo).
    \item Cría y engorde en la dehesa o cebadero: tras el destete, los cerdos pasan a fase de crecimiento. Si son de bellota, se crían en libertad en la dehesa hasta, aproximadamente, los 18 meses, con una fase final de montanera (de octubre a febrero) donde se alimentan de bellotas y pastos. Si son de cebo, pueden criarse en granjas intensivas o semi-extensivas (cebo de campo). En ambos casos, se requieren controles sanitarios (medicación administrada, etc.). La sensórica aquí podría incluir geolocalización GPS de algunos animales en dehesa para asegurar que realmente pastan cierta superficie, o sensores de actividad (existen collares que miden pasos para ganado, podrían aplicarse a cerdos). También puede haber sensores ambientales: estaciones meteorológicas en la dehesa registrando la climatología durante la montanera (que influye en la bellota). Todo esto aportaría al pasaporte digital información sobre el bienestar animal y alimentación: p. ej., número de hectáreas por cerdo (derivado del movimiento), alimentación recibida (certificada por el ganadero). En cebo intensivo, sensores de granja (temperatura nave, calidad del aire) también pueden registrarse. Desde el punto de vista de la trazabilidad, cada lote de pienso que se da a los animales se documenta, y en extensivo se acredita el período de montanera (mínimo 60 días comiendo bellota, con incremento de peso mayor a X kg según norma). Estos eventos (entrada del cerdo a montanera, fin de montanera, etc.) se registran en el sistema y podrían registrarse en blockchain para mayor confianza. En la etapa de cría, se hacen a menudo pesajes periódicos: básculas ganaderas podrían volcar automáticamente el peso a un registro digital cada vez que se pesa un cerdo, combatiendo posible fraude (el peso de entrada/salida de montanera es clave para ver si engordó a base de bellota y en los casos de las DOPs y las IGPs, si cumple con los límites estipulados por Norma).
    \item Transporte al matadero: cuando los cerdos alcanzan el peso adecuado (normalmente  más de 160 kg en bellota), se envían al sacrificio. Aquí la trazabilidad exige documentación del transporte: fecha, hora, matrícula del camión, certificado de limpieza del vehículo, etc. Al llegar, se verifica la identidad de los animales (lectura de crotales) y se asigna un número de lote de matanza. En un DPP, se podría incluir toda esta información como parte de la historia de la pieza, lo cual es relevante para la calidad de la carne.
    \item En el matadero autorizado, se sacrifica al animal siguiendo protocolos sanitarios estrictos. Cada cerdo genera dos piernas (futuros jamones) y dos paletas, además de otros productos. En este punto, se asocia la identificación individual del cerdo con las piezas resultantes. Tradicionalmente, se tatuaba o marcaba el número de lote en la piel de las piernas. Con tecnología moderna, podría hacerse mediante código QR o RFID en las piezas inmediatamente después del faenado para una identificación inequívoca. Tras el faenado, un veterinario oficial realiza la inspección sanitaria; si el cerdo no pasa la inspección, no entra en la cadena alimentaria. Este resultado se registra. Para los aprobados, cada canal se pesa y clasifica. En ibérico, además se toma nota del grado de engrasamiento, que es importante (ej. una característica del bellota es el \textit{toqueado} de grasa infiltrada). Además del resto de datos (número de identificación individual, número de pieza resultante, resultado de la inspección sanitaria), este dato de calidad podría incluirse en el pasaporte digital: por ejemplo, “Grado de grasa dorsal: alto, apto bellota”, pudiendo idenficar lotes de calidad excepcional que podrían venderse como ediciones especiales, ediciones únicas, etc. Lo más relevante es mantener la trazabilidad: esto se logra con etiquetas en cada jamón fresco con un código único (muchos usan códigos de barras pegados). En esta etapa también hay controles de pH de la carne y temperatura post-mortem; estos datos podrían almacenarse en el DPP. Desde el punto de vista DLT, cada jamón tendría ya su “identidad digital” lista para seguirle la pista independiente del resto. Se puede emitir un evento de trazabilidad tipo EPCIS “Transformation Event” que dice: Animal X (ID) -> Jamón Y (ID), Paleta Z, etc., con links a los IDs de piezas. Este es un punto crítico donde el pasaporte digital toma entidad propia para cada jamón.
    \item Salado y postsalado: las piernas frescas pasan a la fase de salazón, donde se cubren en sal marina por 1 día por kilo de peso, aproximadamente. Luego se lavan y equilibran en cámara fría (postsalado) durante semanas. En esta etapa, las variables ambientales son muy importantes: temperatura típicamente de entre 0°C y 5°C y humedad entre el 80\% y el 90\% en salado, luego un poco menos frío en postsalado. Sensores de temperatura y humedad son imprescindibles; muchos secaderos ya monitorizan digitalmente estas cámaras. Para garantizar la calidad, se debe asegurar que cada lote estuvo el tiempo correcto en sal. El pasaporte digital podría contener por lote: “Salazón del 10/2/2025 al 25/2/2025 a 3°C – 85\% HR, sal marina de origen Alicante”. Incluso se podría integrar información sobre la sal usada (hay normas que requieren que sea sal marina sin aditivos, etc.), integrando el pasaporte digital de la sal marina. Un sensor IoT en la sala de salazón podría registrar cada hora esas condiciones y anclarlas a la cadena; en caso de desviación (por ejemplo, si la temp subió por encima de 5°C), ese registro queda y sirve para auditoría. Tras el salado, en la fase de reposo en cámara, la pierna va perdiendo agua lentamente. Podría medirse peso de piezas antes y después del salado; esas mediciones se apuntan en control de calidad y podrían subirse al DPP para reflejar la evolución (pérdida de X\% de peso tras salado, normal).
    \item Secado y maduración (curado): Eesta es la etapa más larga y característica: los jamones se cuelgan en secaderos naturales o cámaras, primero a temperaturas suaves primaverales (15-20°C) y luego pasan a bodegas frescas (verano-invierno siguiente) donde maduran a 18-25°C con humedades decrecientes. El proceso completo dura mínimo 14 meses en serranos, pero en ibéricos de bellota se alarga a 24, 36 o más meses. Durante este tiempo, las condiciones ambientales son críticas: tradicionalmente se usaban ventanas para regular, hoy en el ámbito industrial se usan controles climáticos automáticos. Sensores de temperatura, humedad y circulación de aire se despliegan por todo el secadero. Un sistema IoT recogería miles de datos de clima a lo largo de los años de curación. No se registrarían todos en el pasaporte por practicidad, pero sí se pueden generar indicadores clave: por ejemplo, número de días que el jamón estuvo dentro de ciertos rangos de temperatura, o perfiles estacionales. Un enfoque interesante es colocar sensores electrónicos dentro de piezas piloto (hay investigaciones con sensores de humedad internos en jamones para estudiar el secado); esos podrían aportar datos de difusión de sal, etc., aunque eso sería más para I+D que para cada pasaporte de consumo. En cuanto a controles de calidad en esta fase: maestros jamoneros realizan el \textit{calado} (pinchar con hueso de caña para oler) para comprobar aroma. Ese resultado sensorial (apto o no apto) podría digitalizarse como un evento: QualityInspectionEvent – aroma OK en mes 18. Además, se realizan movimientos internos: se suele cambiar las piezas de sitio, moverlas de planta, etc. Cada movimiento (por lotes) dentro del secadero puede registrarse para mantener orden. Un DPP podría reflejar “Curado: X meses en secadero natural de Jabugo (Huelva), altitud 650m)”, poniendo en valor características geográficas que interesan al marketing también. Sensores externos como estaciones meteorológicas simples podrían registrar cómo fueron las condiciones climáticas reales en esos años para aquellos secaderos naturales que dependen del clima exterior. Esos datos podrían asociarse al pasaporte para verdaderamente contar la historia del jamón: por ejemplo, “año de secado con verano caluroso, que potenció la sudación de la grasa”. Desde la perspectiva de trazabilidad formal, durante la curación no se mezclan lotes, es más bien un almacenamiento controlado. Sin embargo, existe merma (al final se pierde un 35\% del peso inicial). Sería interesante registrar el peso final de cada jamón al descolgar, porque indica concentración de sabor; un pasaporte digital podría listar: peso inicial en comparación con peso final, porcentaje de merma. Ese dato hoy se usa internamente para clasificación, pero no llega al consumidor – en un DPP podría ser un sello de calidad (ej. “curación lenta, perdió 34\% de su peso”).
    \item Clasificación final, etiquetado y precintado: tras la maduración, cada jamón ibérico es evaluado para su categoría de venta. En el jamón ibérico, aquí es donde se coloca el precinto de color de la Norma (negro, rojo, etc.) si no se puso antes. Según la norma RD 4/2014, el precinto obligatorio debe colocarse antes de la salida del jamón de la industria y va asociado a la trazabilidad del animal \cite{ministerio_de_agricultura_alimentacion_y_medio_ambiente_real_2014}. En la práctica, la mayoría de industrias coloca el precinto al inicio de curación o incluso en fresco. Pero supongamos que se hace al final para confirmar que cumplió todos los requisitos. El precinto lleva un código (código de barras o datamatrix) que identifica la pieza en la base de datos ÍTACA. Aquí la APP Ibérico permite a cualquiera escanear ese código y ver información básica: raza, alimentación, fecha de sacrificio, empresa elaboradora \cite{noauthor_app_nodate}. Este es ya una especie de pasaporte digital centralizado actual. Al etiquetar, también se añade la etiqueta comercial con marca, lote, peso, y a veces un código QR de la empresa que apunta a su web o certificado blockchain privado (algunas marcas han implementado sus propios códigos de autenticidad). Es crucial en esta etapa que toda la información recopilada se sincronice: la base de datos interna de la empresa, ÍTACA, y eventualmente la blockchain/DPP deben concordar. Un posible procedimiento es generar automáticamente el registro en la DLT en cuanto se asigna un precinto a una pieza, cerrando así la trazabilidad con un evento final Object Event: Jamón \#XYZ precintado como Bellota 100\% Ibérico. Sensórica aquí no aplica mucho, pero sí se podría hacer una foto 360° del jamón y subirla al pasaporte digital (quizá almacenada IPFS y hash en blockchain) como verificación visual – ya existen apps que con visión artificial autentican jamones por su aspecto, un DPP podría incorporarlo en el futuro.
    \item Distribución y venta: Los jamones terminados se envían al mercado (ya sea a distribuidores, tiendas gourmet, exportación, etc). En esta fase logística, la trazabilidad legal continúa en forma de documentos de envío (albaranes con lotes). Un pasaporte digital puede registrar la primera salida comercial: por ejemplo, “Jamón \#XYZ enviado el 10/10/2025 a Almacén Madrid para distribución”. Si se quisiera afinar más, se podrían equipar contenedores de jamones con sensores de temperatura si van en rutas largas, para asegurar que no superen cierto umbral (aunque el jamón curado no requiere frío estrictamente, temperaturas muy altas podrían dañar la calidad). De cara al consumidor, es interesante saber la fecha de expedición y quizá el tiempo que ha pasado desde que salió de bodega hasta que lo compra, por frescura del corte (los jamones enteros pueden conservarse mucho tiempo, pero envasados loncheados tienen fecha de consumo preferente). Otro aspecto relevante a tener en cuenta son las exportaciones, donde se requiere a veces certificados sanitarios adicionales (p.ej. China exige que la pieza pase 30 días a -20°C si no está cocida, para eliminar riesgo triquina). Si esto ocurre, debería reflejarse: “congelación preventiva realizada del 01/06/2025 al 30/06/2025”. Un sensor de temperatura podría validar que se cumplió esa congelación. Esto dotaría al pasaporte de datos de seguridad alimentaria en mercados lejanos. Cuando el jamón llega a un punto de venta, se podría registrar un evento de recepción en el sistema del agente minorista, integrando la trazabilidad B2B con la DLT. Realísticamente, en una primera implementación, la trazabilidad DLT podría detenerse al salir de la fábrica, y luego retomar con el escaneo por el consumidor final.
    \item Postventa y consumo: una vez en manos del consumidor (ya sea un particular que compra un jamón entero, un restaurante, etc.), la vida técnica del producto termina con su consumo. Sin embargo, el concepto de pasaporte digital debe extenderse a esta fase para fines de retroalimentación y circularidad. Por ejemplo, el pasaporte (vía un código QR en la etiqueta) puede permitir al consumidor registrar que consumió el producto y dar su feedback (calificación de sabor, etc.), lo cual la empresa podría captar para control de calidad. También, como mencionamos, puede guiar en la disposición final (reciclaje de envases). Incluso podría incentivar la reutilización: en productos textiles hay pasaportes que fomentan reventa de segunda mano; en jamón no aplica la reventa, pero podría aplicar programas de recogida de restos (por ejemplo, una marca podría ofrecer recoger el hueso a cambio de algún descuento, para aprovecharlo industrialmente). Son ideas quizás no maduras comercialmente, pero el DPP abre la puerta a innovaciones en la relación postventa. Desde un punto de vista distinto, si pensamos en la trazabilidad en caso de incidencias: si hubiera que hacer un recall o retirada (por ejemplo, se descubre que cierta partida de sal estaba contaminada), tener cada pieza identificada digitalmente hasta el consumidor facilitaría notificarles. Un sistema DPP podría enviar alerta (si el consumidor registró su producto escaneándolo) de “no consuma el producto X”. Esto se alinea con los objetivos de seguridad alimentaria mejorada. Por último, en postconsumo, se podría emitir un “fin de vida” en el pasaporte cuando el producto es consumido o descartado, aunque es más relevante en productos duraderos que en alimentos. En general, tener un pasaporte digital extiende la cadena de valor hasta el consumidor final informado, convirtiendo la tradicional trazabilidad B2B (del corral a la tienda) en una trazabilidad también B2C (hsata la mesa del consumidor).
\end{enumerate}

A lo largo de todas estas etapas, existe documentación oficial que define requerimientos y buenas prácticas: la Norma de Calidad del Ibérico (RD 4/2014) ya citada, las guías de trazabilidad de ASICI, manuales del MAPA sobre inspección de jamones, etc. Por ejemplo, ASICI con su sistema ÍTACA ha logrado identificar con precinto más de 7,15 millones de jamones ibéricos en 2022, lo que muestra la escala de datos manejados (cada uno con su historial). Asimismo, las 4 DOPs ibéricas añaden control extra a  unos 256.000 jamones anuales \cite{asociacion_interprofesional_del_cerdo_iberico_asici_despierta_2023}. Toda esta documentación (registros de campo, guías de prácticas, normas de calidad) sería la base sobre la que el pasaporte digital extraería datos. El DPP no viene a reinventar la trazabilidad, sino a unificarla y enriquecerla: consolidar en un solo recurso digital datos actualmente dispersos en etiquetas, precintos, albaranes y certificados, y presentarlos de forma coherente y verificable. De este modo, cada capítulo de la vida del jamón (desde la encina de la dehesa hasta el plato) queda reflejado y respaldado por evidencias en la cadena de bloques, alineándose con el enfoque de \textit{del campo a la mesa} promovido por la UE \cite{noauthor_farm_nodate}.

\subsection{Sostenibilidad en la Industria del Jamón Ibérico}
El sector del jamón ibérico se ha alineado firmemente con los objetivos de sostenibilidad de la UE. De hecho, el propio sector se ha adherido al Pacto Verde Europeo («Green Deal») y a sus compromisos de lograr un impacto climático neutro en 2050, avanzando hacia un sistema alimentario más sostenible. Este compromiso se refleja en múltiples iniciativas ambientales, como el aprovechamiento de la dehesa (ecosistema mediterráneo donde se cría el cerdo ibérico) como sumidero de CO\textsubscript{2}, la reducción de emisiones de gases de efecto invernadero y la compensación de huella ecológica mediante reforestación. Asimismo, la industria está implementando medidas de economía circular, con uso eficiente de recursos, planes de gestión de residuos (e.g., envases más ecológicos y disminución de plásticos de un solo uso) y reducción del consumo de agua (optimización hídrica con hasta un 30\% menos de uso en los últimos años.

En línea con las estrategias europeas (“De la Granja a la Mesa”, biodiversidad, clima, etc.), el modelo extensivo del ibérico en la dehesa presenta ventajas sostenibles: promueve el bienestar animal (cría al aire libre con alimentación natural) y contribuye al desarrollo rural. El sector ibérico actúa como motor socioeconómico en zonas rurales, fijando población y generando empleo local, en concordancia con los Objetivos de Desarrollo Sostenible de la ONU.

Todas estas iniciativas sitúan al ibérico en sintonía con las políticas del \textit{European Green Deal}, que busca una Europa climáticamente neutra, circular y justa. La propia Comisión Europea, en el Pacto Verde Europeo de 2019, recalcó la necesidad de transformar la economía hacia la neutralidad climática al 2050 y destacó que los productos sostenibles (incluidos los agroalimentarios) desempeñan un papel esencial en esa transición.
En suma, la industria del jamón ibérico está incorporando la sostenibilidad medioambiental, social y económica en su cadena de valor, preparándose para las exigencias presentes y futuras de la UE en materia de impacto ambiental, trazabilidad y economía circular.

\section{Definición del Problema, Objetivos del Trabajo y Alcance}\label{sec:intro-definición}
La cadena de valor del jamón ibérico opera con múltiples fuentes de datos heterogéneas y con obligaciones de trazabilidad que, si bien se cumplen, dificultan la verificación ágil de autenticidad y condiciones de proceso a nivel de pieza. Esta dispersión limita la transparencia hacia consumidores y distribuidores internacionales y complica auditorías rápidas ante incidencias o sospechas de fraude \cite{}.

El objetivo general del proyecto persigue diseñar y prototipar un \textit{blueprint} de DPP (Pasaporte Digital de Producto, por sus siglas en inglés) para jamón ibérico, alineado con el marco ESPR y estándares GS1, que integre datos \textit{on/off-chain} y ofrezca vistas diferenciadas por rol, empleando IOTA Tangle como capa de inmutabilidad \cite{}

Los objetivos específicos son:
\begin{itemize}
\item Modelar un esquema mínimo del DPP para jamón ibérico (identificación GS1, eventos EPCIS, metadatos de calidad y certificación) y su control de accesos por rol \cite{}.
\item Implementar un prototipo funcional del DPP (ingesta de eventos simulados, anclaje en IOTA Tangle y \textit{viewer} web con QR basado en roles para vista pública,operativa,auditor) \cite{}.
\item Evaluar beneficios operativos: reducción del TIE, integridad frente a manipulaciones \textit{off-chain} y esfuerzo de adopción para pymes (CapEx/OpEx) \cite{}. \textbf{(EXCLUIRLO?? NO VEO FACILIDAD DE MEDIR EL TIE Y NO ES COMÚN LA RETIRADA DE PRODUCTOS EN EL JAMÓN IBÉRICO)}
\item Delimitar implicaciones regulatorias (ESPR, RGPD, RD 4/2014, DOP/IGP) y compatibilidades con sistemas sectoriales (ÍTACA) \cite{}.
\end{itemize}

El prototipo cubre todas las categorías comerciales del jamón ibérico según la RD 4/2014; abarca desde la cría y engorde hasta los puntos de clasificación, precintado y expedición, e incorpora circularidad postventa a nivel conceptual. Se emplean datos simulados realistas y se prioriza sensórica de cámaras de curado y eventos documentales clave; quedan fuera despliegues masivos de geolocalización animal y visiones por computador in situ (se consideran como líneas futuras). El enfoque es sectorial y extensible a DOP/IGP como capas adicionales \cite{}.

\section{Metodología}\label{sec:intro-metodologia}
El trabajo se condujo en cuatro fases iterativas:

\textit{Fase I — Marco y requisitos.} Revisión normativa (ESPR, RD 4/2014, RGPD), estándares GS1 (EPCIS 2.0, Digital Link) y sistemas sectoriales (ÍTACA). Derivación de requisitos funcionales y no funcionales del DPP y de sus vistas por rol \cite{}.

\textit{Fase II — Modelo de datos.} Definición del identificador resoluble (GTIN+Lote/Serie), del diccionario de campos mínimos y de los tipos de evento EPCIS para cada etapa (salazón, post-salado, curado, clasificación/precintado, expedición). Diseño de perfiles de acceso y políticas de \textit{data minimization} \cite{}.

\textit{Fase III — Prototipo.} Desarrollo de \textit{scripts} en Python, anclaje en IOTA Tangle mediante \textit{client SDK}, almacenamiento \textit{off-chain} de detalle y visor QR (vista pública y operativa). Generación de datasets simulados con variabilidad realista (temperatura y humedad relativa en cámaras; pesos y mermas) \cite{}.

\textit{Fase IV — Evaluación.} Pruebas de integridad (detección de desalineaciones entre eventos y anclajes), estimación del TIE, análisis de sensibilidad de costes de adopción y discusión de riesgos y limitaciones.

\section{Limitaciones, Hipótesis de Partida y Estructura de la Memoria}\label{sec:intro-limitaciones}
En cuanto a las limitaciones de la memoria, el trabajo se acotó por disponibilidad de datos y alcance técnico. No se dispuso de acceso a bases de datos industriales ni a integraciones en producción con sistemas sectoriales (ÍTACA), por lo que se emplearon datasets simulados con rangos y variabilidad coherentes con documentación pública y normativa aplicable \cite{}. La instrumentación IoT se redujo a supuestos verosímiles de sensórica ambiental en cámaras (temperatura y humedad) y a registros documentales clave (fechas, pesos, precinto), sin desplegar geolocalización masiva de animales ni visión artificial en línea. Los datos de alto volumen y <<granularidad fina>> se almacenaron \textit{off-chain} y se anclaron mediante resúmenes criptográficos en IOTA Tangle, priorizando integridad y eficiencia \cite{}. %La evaluación de sostenibilidad no abordó un ACV (Analisis de Ciclo de Vida) completo sectorial;
Se definieron campos y pautas compatibles con el Pacto Verde y con el DPP europeo para su progresiva incorporación \cite{}. La comparación exhaustiva entre DLTs no formó parte del alcance. La verificación de cumplimiento RGPD se limitó al diseño de capas de acceso y minimización de datos personales, sin auditorías legales externas \cite{}.

Como hipótesis de partida, se asumió que el DPP del jamón ibérico debía alinearse con el marco europeo del DPP (ESPR) y con estándares GS1 (EPCIS 2.0 y Digital Link) para garantizar interoperabilidad y escalabilidad \cite{}. Se consideró que las obligaciones de trazabilidad alimentaria vigentes persistían en sus sistemas de origen (ÍTACA, ERPs), y que el DPP actuaría como capa integradora y verificable, sin sustituir los registros establecidos por la norma \cite{}. Se planteó que la gobernanza de identificadores y resoluciones podría recaer en una entidad interprofesional (ASICI), posibilitando un modelo de servicios comunes y minimizando cambios para pymes. Se adoptó que la sensórica prioritaria para el prototipo correspondía a cámaras de salazón, post–salado y secado/maduración; que el consumidor consultaría el DPP mediante QR (GS1 Digital Link) con vista pública; y que operadores, auditores y autoridades accederían a vistas ampliadas controladas por permisos y credenciales verificables \cite{}. Se contempló que el reporte REACH sería relevante principalmente para materiales en contacto con alimentos y para ingredientes auxiliares (sal y piensos), más que para el producto cárnico en sí \cite{}. Se asumió que la circularidad posconsumo tendría tratamiento conceptual y campos ampliables, sin exigir la involucración del consumidor final.

La memoria se estructuró para conducir desde el contexto sectorial y normativo hasta el diseño, la implementación y la evaluación de la propuesta:
\begin{itemize}
    \item \textbf{Capítulo 1. Introducción}: antecedentes, diferencias entre jamón ibérico y serrano, cadena de suministro y de valor, definición del problema, objetivos, metodología, limitaciones e hipótesis.
    \item \textbf{Capítulo 2. Marco normativo}: ESPR y DPP, trazabilidad alimentaria obligatoria y voluntaria, RGPD, REACH, Norma de Calidad del Ibérico (RD 4/2014), DOP/IGP como capas adicionales, y encaje con sistemas sectoriales (ASICI–ÍTACA).
    \item \textbf{Capítulo 3. Estado del arte}: fundamentos de DLT/IoT y foco en IOTA Tangle como infraestructura de inmutabilidad y direccionamiento de datos.
    \item \textbf{Capítulo 4. Diseño del DPP}: requisitos funcionales y no funcionales, modelo de datos (EPCIS 2.0 y GS1 Digital Link), roles y controles de acceso, y mapeo por etapas del proceso.
    \item \textbf{Capítulo 5. Implementación}: entorno, simulación de datos, anclaje en IOTA, API de ingestión y visor QR con vistas por rol.
    \item \textbf{Capítulo 6. Evaluación y análisis}: pruebas de integridad y coherencia, estimación del tiempo de investigación de expedientes, escalabilidad/interoperabilidad, y análisis económico (CapEx/OpEx) con sensibilidad para pymes.
    \item \textbf{Capítulo 7. Conclusiones}: principales hallazgos, aportación y limitaciones.
    \item \textbf{Capítulo 8. Planificación y presupuesto}: EDT (Estructura de Desglose de Trabajo), cronograma y costes de adopción.
    \item \textbf{Capítulo 9. Ética}: consideraciones éticas en datos y transparencia.
    \item \textbf{Capítulo 10. ODS}: contribución del DPP a objetivos de sostenibilidad.
    \item \textbf{Capítulo 11. Líneas futuras}: ampliaciones técnicas (integración real con ÍTACA, ACV sectorial, credenciales verificables para roles) y extensión a otros curados.
\end{itemize}

Este esquema aseguró coherencia con el marco regulatorio europeo y español, facilitó el mapeo entre etapas del proceso y el modelo de datos del DPP, y permitió evaluar la viabilidad técnica y económica de la solución propuesta en un contexto realista del sector ibérico.