\chapter{EVALUACIÓN Y ANÁLISIS}\label{ch:evaluacion}

\section{Evaluación Técnica: Rendimiento, Escalabilidad, Seguridad e Interoperabilidad}\label{sec:evaluacion-tecnica}

\section{Evaluacion Económica: Estimación de costes y Análisis de Sensibilidad}\label{sec:evaluacion-economica}

\subsection{Costes y viabilidad económica para pymes jamoneras}
Implementar un sistema de Pasaporte Digital de Producto basado en Blockchain e IoT conlleva tanto inversiones iniciales (CAPEX) como costes operativos continuos (OPEX), que es fundamental estimar, especialmente pensando en pequeñas y medianas empresas del sector jamón ibérico. A continuación, se analizan los principales rubros de coste, barreras de entrada y estrategias para hacer viable la adopción progresiva:
\begin{itemize}
	\item CAPEX (Inversión inicial): Incluye la compra o desarrollo de hardware, software e infraestructura necesaria para poner en marcha el sistema. Entre los elementos destacables:
    \begin{itemize}
        \item Dispositivos IoT y sensores: por cada etapa donde se monitorearán variables, la empresa debe adquirir sensores (de temperatura, humedad, GPS, etc.) y posiblemente data loggers o gateways (como Raspberry Pi) para transmitir esos datos. Afortunadamente, muchos sensores son relativamente económicos hoy: un sensor de temperatura/humedad industrial puede costar del orden de 50-100€, un dispositivo GPS robusto también 100€, y una Raspberry Pi 4 completa unos 80€ con accesorios. Si supongamos una explotación pequeña instala 5 sensores en secadero, 2 en cámaras de salado y 1 estación meteorológica, más un gateway, la inversión podría rondar 1.000-2.000€ en hardware IoT. Para un secadero mediano, probablemente más sensores, digamos 5.000€.
        \item Equipo de nodo blockchain: si se opta por operar un nodo propio (un nodo IOTA) se necesita un servidor que puede ser modesto (una RPi o un mini PC). Un pequeño servidor local podría costar 500-1000€. También puede usarse infraestructura en la nube (en cuyo caso CAPEX baja pero aparece OPEX).
        \item Desarrollo o adquisición de software: aquí puede estar el mayor coste inicial ya que hay que desarrollar la integración entre sensores y blockchain, la interfaz del pasaporte digital, etc. Una pyme puede no tener programadores internos, por lo que tal vez contrate a un proveedor. Los costes de desarrollo software dependen del alcance (si se usa una solución existente tipo SaaS, podría ser menor). Existen ya startups ofreciendo trazabilidad blockchain como servicio a alimentarias a precios accesibles (una plataforma blockchain de trazabilidad alimentaria como QualityChain ofrece suscripciones anuales desde 249€ sin costes iniciales elevados \cite{salaris_trazabilidad_2020}, enfocada a microempresas, donde la empresa básicamente introduce los datos manualmente y ellos proveen el backend en blockchain; sin embargo, para una solución IoT integrada personalizada, un desarrollo a medida podría costar varios miles a decenas de miles de euros). Un TFG desarrollando un prototipo minimiza ese coste al ser un estudio, pero para producción comercial habría que considerar la inversión en un software robusto, app móvil, etc.
        \item Capacitación y adecuación de procesos: implementar DPP implica formar al personal en nuevas herramientas, cambiar algunas rutinas ( escanear códigos en matanza, ingresar datos en un sistema digital en campo). Esa capacitación tiene un costo (tiempo, potencialmente contratación de especialistas para entrenar). Aunque puede no ser un gasto monetario directo, sí un esfuerzo que la empresa debe invertir sus recursos (trabajadores, tiempo).
    \end{itemize}
    \item OPEX (Costes operativos continuos): aquellos recurrentes para mantener el sistema funcionando. Tales como:
    \begin{itemize}
        \item Mantenimiento de sensores y redes IoT: los sensores requieren calibración periódica (especialmente aquellos de precisión como los de clima), baterías si son inalámbricos, o recambios si fallan. También la conectividad: una dehesa puede necesitar una tarjeta SIM IoT para enviar datos vía GSM, con coste mensual (aunque mínimo, quizá 5-10€/mes por SIM M2M). Si hay 5 SIM, son 50€ mes. En total, mantenimiento y comunicaciones IoT tal vez unos pocos cientos de € al año para una pyme pequeña.
        \item Hosting de nodos y datos: si la empresa mantiene un nodo blockchain en la nube (por ejemplo, un VPS para un nodo IOTA), el coste podría ser 20-50€ al mes. Si se usa un servicio externo (QualityChain u otros), estos cobran suscripciones. Hemos visto suscripciones base 249€/año para microempresas por una solución simple \cite{salaris_trazabilidad_2020}. Soluciones más completas o usuarios medianos podrían ser del orden de 1000-2000€/año. Por ejemplo, IBM Food Trust (blockchain permisionada para alimentación) tiene un modelo de pago por participación, que para pequeños productores se cifraba en unos pocos cientos de euros mensuales en el pasado.
        \item Soporte técnico y actualizaciones: aquí entra la necesidad de resolver problemas, actualizar el software conforme evoluciona la blockchain (IOTA sacará versiones nuevas, hay que actualizar nodos), y mejorar la solución con el tiempo. Si la empresa no tiene IT interno, puede contratar un servicio de soporte (posiblemente el mismo proveedor SaaS). Esto podría ser un fee anual o por demanda. Una cifra conservativa seríau un 15\% de la inversión de software por año en mantenimiento evolutivo es usual en TI.
        \item Costes de transacción: en IOTA no hay comisiones.
    \end{itemize}
\end{itemize}

\subsection{Barreras de entrada para PYMEs}
El sector jamonero ibérico incluye grandes empresas muy avanzadas tecnológicamente, pero también muchas explotaciones familiares y secaderos tradicionales de pequeño tamaño. Las principales barreras para estos últimos son:
\begin{itemize}
    \item Desconocimiento y falta de personal especializado: Blockchain e IoT son conceptos ajenos a muchos productores rurales. Puede haber resistencia al cambio tecnológico o miedo a no saber manejarlo y contratar especialistas es caro.
    \item Costo inicial y ROI incierto: aunque los costos señalados no son astronómicos, para una explotación pequeña invertir, digamos, 10.000€ en un sistema digital puede ser difícil de justificar si no ve un retorno claro (ya cumplen trazabilidad legal con métodos existentes). El ROI del DPP vendría por mejoras de eficiencia (automatizar registros), evitar sanciones, y posiblemente poder vender mejor el producto por diferenciación. Este último punto es clave: si el mercado no paga más por un jamón con DPP, el productor no tiene incentivo financiero inmediato. Convencerlos de que es una inversión a futuro (porque la UE lo exigirá, o porque mejorará su marca) es un desafío.
    \item Infraestructura de internet deficiente: en zonas rurales (dehesas, secaderos en pueblos pequeños) la conectividad a internet puede ser limitada. Esto dificulta la implementación IoT que usualmente requiere enviar datos a la nube o al blockchain.
    \item Interoperabilidad y estándares: Una pyme podría temer quedar “atrapada” en una solución propietaria: si adoptan la plataforma X y luego la UE exige la plataforma Y, tendrían que migrar. Por eso es importante basarse en estándares abiertos (GS1, etc.) para que la inversión sea reutilizable.
    \item Cultura y hábitos: el sector ibérico combina tradición artesanal con tecnología; lgunos maestros jamoneros pueden ver con recelo sensorizar todo, prefieren su olfato y tacto. Hay que integrar la tecnología sin menoscabar el saber hacer, sino apoyándolo.
\end{itemize}

\subsection{Análisis de sensibilidad y beneficios potenciales}
Es útil contemplar distintos escenarios de adopción y sus impactos económicos:
\begin{itemize}
    \item Escenario conservador: Solo se implementa un “primer nivel de trazabilidad digital”, es decir, la empresa introduce manualmente datos clave en una plataforma blockchain (sin muchos sensores IoT). Esto, como explica QualityChain, equivale a un storytelling asegurado en blockchain, más que trazabilidad granular \cite{salaris_trazabilidad_2020}. El coste es bajo (suscripción anual pequeña) y el beneficio es sobre todo de marketing (transparencia). Muchas pymes podrían empezar por aquí. Por ejemplo, una microempresa contrata un plan de 300€/año y dedica 1h a la semana a subir datos de sus lotes manualmente. La sensibilidad: con vender 2-3 jamones más gracias a esa diferenciación, ya se paga el servicio.
    \item Escenario medio: La empresa mediana invierte en sensores IoT y automatiza parcialmente. Esto supone CAPEX en equipos y un coste de integración de software. Sin embargo, la ganancia aquí no es solo marketing sino también eficiencia interna: al tener datos en tiempo real de secaderos, puede optimizar energía, reducir mermas, detectar problemas antes (lo que ahorra dinero). Por ejemplo, si un sensor avisa de humedad fuera de rango y previene un lote mohoso, se ahorraron miles de euros en producto. Difícil de cuantificar, pero real. Además, automatizar registros ahorra horas de personal llenando papeles. Si un operario costaba 15€/h y dedicaba 10h mensuales a papeleo de trazabilidad, automatizarlo le ahorra 150€ al mes, 1800€/año. Ese ahorro ayuda a amortizar la inversión en TI.
    \item Escenario avanzado: Consorcio de grandes productores implementa DPP completo e integran su cadena. Aquí los costes son mayores (desarrollo consorcial, nodos, etc.), pero las economías de escala también juegan: podrían compartir infraestructura y gastos. Además, un consorcio puede tener más fácil acceder a subvenciones (la UE y Gobiernos están apoyando digitalización agroalimentaria). Con la obligación de DPP en el horizonte, es plausible que haya fondos de la UE para que sectores como el cárnico adopten estos sistemas. Si una pyme accede a subvención del 50\% de los costes, la viabilidad sube notablemente.
\end{itemize}

Por otra parte, dada la barrera de coste/experiencia, una estrategia que minimiza impactos sería la implementación por fases:
\begin{itemize}
    \item Fase 1: Digitalización básica de la trazabilidad existente. Reemplazar registros en papel por registros electrónicos (tablet o PC) conectados a un sistema central (que puede no ser blockchain aún). Esto ya reduce errores y facilita luego volcar a DPP. Por ejemplo, que el ganadero en dehesa use una app móvil para registrar montanera en vez de una hoja de cálculo.
    \item Fase 2: Piloto de blockchain con unos pocos datos clave. Seleccionar, por ejemplo, 50 jamones de bellota y hacerles un pasaporte digital con la información más relevante (nacimiento, montanera, sacrificio, curación, etc.), para probar la tecnología y recabar la respuesta de los consumidores de esos productos. Esta fase con un alcance limitado permite afinar el sistema sin arriesgar toda la producción.
    \item Fase 3: Extensión a toda la producción y añadido de IoT progresivo. Una vez comprobado el valor, ir incorporando más sensores: primero quizá en secadero (donde es fijo y fácil de implementar), luego en la dehesa para los ganaderos que quieran (podría ser voluntario al inicio), etc. También incluir más campos de datos en el pasaporte digital gradualmente (empezar con raza y alimentación, luego añadir huella de carbono calculada, luego añadir info nutricional, etc.).
    \item Fase 4: Integración completa en operaciones: llegar a un punto en que casi no haya duplicación, es decir, los datos que ya se recogen para cumplir la norma (precintos, inspecciones) se integran automáticamente al DPP sin esfuerzo adicional. Esto requerirá convencer a autoridades (que se acepten reportes digitales en lugar de ciertos papeles).
\end{itemize}

Esta progresión minimiza el choque inicial y reparte la inversión en el tiempo, permitiendo reinvertir beneficios obtenidos.

Un resumen de análisis financiero simplificado para una pyme jamonera podría verse así:
\begin{itemize}
    \item Inversión inicial 10.000€ (sensores + software).
    \item Costes anuales 2.000€ (soporte, hosting, comunicaciones).
    \item Beneficios anuales tangibles:
    \begin{itemize}
        \item Ahorro de tiempo y mejora de eficiencia 1.000€.
        \item Marketing/trust intangible (difícil de cuantificar, pero supongamos aumenta ventas en 2\%). Si facturaba 200k€/año, 2\% son 4.000€ más ventas, potencialmente.
    \end{itemize}
\end{itemize}

Esto haría que empezase a compensar. Con economías de escala y madurez, los costes de tecnología tenderán a bajar, mientras que el valor de la confianza digital tenderá a subir (por normativas y preferencias).

En cuanto al impacto en precio de producto: Es posible que en el futuro cercano, los jamones con DPP obtengan mejor aceptación de clientes (especialmente en exportación y retail \textit{premium}). Podrían venderse con un pequeño margen adicional al destacar su trazabilidad verificada. Si un jamón de 500€ logra venderse a 520€ por llevar un pasaporte digital que demuestra su excelencia, esos 20€ extra son beneficio que justifica la inversión digital. Por ahora es hipotético, pero con consumidores cada vez más atentos a transparencia, por lo que es plausible. Grandes distribuidores (supermercados) también presionan para mayor trazabilidad; podrían requerir DPP a proveedores en unos años, y quien no lo tenga podría perder mercado. Así, otro coste, intangible y difícil de calcular, es el coste de no adoptar la tecnología. Si la UE en 2030 dice que todos los productos cárnicos deben tener DPP (escenario posible), quien se haya adelantado tendrá ventaja, quien no, deberá invertir contrarreloj, lo que se traduce en que adoptar progresivamente ahora reduce riesgos futuros.

Para la viabilidad, es previsible que se cuente con apoyos, especialmente apoyo institucional: la UE a través de fondos NextGen promueve la digitalización agroalimentaria; en España (iniciativas de AceleraPyme, KIT Digital, etc.) podrían subvencionar a pequeñas empresas la implantación de sistemas de trazabilidad blockchain \cite{acelerapyme_mejora_2023}. También ASICI, como interprofesional, podría implementar su propia plataforma blockchain de uso común, diluyendo costes individuales. Si, por ejemplo, ASICI integra blockchain en ÍTACA y lo ofrece a todos sus asociados, las pymes no tendrían que hacer casi nada aparte de usar la plataforma renovada. Este modelo consorcial reduciría las barreras (coste individual casi nulo, ya cubierto por cuotas interprofesionales).

Por otra parte, modelos de negocio podrían surgir para facilitar la viabilidad: Proveedores tecnológicos podrían ofrecer soluciones a éxito o por unidad de producto. Imaginemos un modelo: \textit{“x céntimos por jamón trazado”; i me cobran 0,10€ por jamón para mantener su pasaporte digital, eso serían 10€ por 100 jamones, muy asumible si me da valor}. O un esquema freemium: gratuito para los primeros 100 jamones, de pago a partir de ahí. Estas fórmulas bajarían la barrera de coste inicial.
De hecho, QualityChain y otras plataformas modulares permiten escalar la complejidad según el tamaño de la empresa \cite{salaris_trazabilidad_2020}.

La viabilidad económica del Pasaporte Digital para jamón ibérico requerirá probablemente una combinación de reducción de costes tecnológicos (lo cual está ocurriendo naturalmente), soporte compartido o institucional, y demostración de valor (tanto en eficiencia interna como en ventajas comerciales). Empezar pequeño, mostrando \textit{rápidas victorias}, ayudará a convencer a más actores y generar un efecto red: cuantos más productores adopten, más estándar se vuelve y más presión habrá para todos (lo cual nivela la competencia).

Dado que la UE apunta a hacer obligatorios estos pasaportes digitales por sostenibilidad en diversos sectores \cite{noauthor_communication_2025}, es muy probable que a medio plazo haya también un mandato para ciertos alimentos (por ejemplo, podría iniciarse con alimentos con indicación geográfica, como un proyecto piloto). Estar por delante de esa curva regulatoria hará que las empresas incurran en menos costes de última hora y puedan incluso influir en cómo se configuran esos estándares sectoriales.

Por eso, la adopción progresiva y proactiva es la estrategia sensata para la industria, convirtiendo un posible cumplimiento futuro en una oportunidad presente de innovación y liderazgo en trazabilidad.

\section{Evaluación de Sostenibilidad y Circularidad: Huella de Carbono y Contribución al Pacto Verde}\label{sec:evaluacion-sostenibilidad1}

\section{Sostenibilidad y Circularidad en la Industria Ibérica: Huella de Carbono, Análisis del Ciclo de Vida}\label{sec:evaluacion-sostenibilidad2}
\subsection{Impacto ambiental a lo largo de la cadena de valor}
La producción de jamón ibérico abarca múltiples etapas – desde la cría del cerdo en la dehesa hasta la distribución del producto curado – y cada una conlleva cargas ambientales específicas. Estudios de Análisis de Ciclo de Vida (ACV) recientes han cuantificado estos impactos.

En términos de huella de carbono, la fase de ganadería (crianza del cerdo) es, con diferencia, la más intensa: aproximadamente un 78\% de las emisiones de GEI asociadas a un jamón provienen de la cría del animal (incluyendo la producción de su alimento, pastoreo, fermentación entérica menor que en rumiantes, y gestión del estiércol). La siguiente etapa más significativa es el procesado/curado en bodega o secadero, que aporta cerca del 22\% de las emisiones – esto se debe al consumo energético prolongado (años de curación con controles de temperatura/humedad, ventilación). En cambio, el sacrificio y faenado en matadero añade un impacto marginal (del orden del 0,4\% de la huella total), al ser un proceso rápido y relativamente eficiente.En total, producir un jamón curado puede emitir del orden de 2,4 kg de CO\textsubscript{2} por kilogramo de jamón: por ejemplo, la DOP Jamón de Teruel calculó 21,87 kg CO\textsubscript{2} por pieza de 9 kg \cite{noauthor_producir_nodate}.

En jamones ibéricos de bellota, la huella puede variar según el manejo, pero tiende a ser menor que la de cerdos blancos intensivos debido a ciertas ventajas del sistema extensivo \cite{noauthor_huella_2025}. En la dehesa, donde el cerdo ibérico pasta durante la montanera, el ecosistema actúa como sumidero de carbono: las encinas y alcornoques fijan el CO\textsubscript{2}, el suelo retiene carbono orgánico, y el pastoreo bien gestionado puede mejorar la biodiversidad. Esto compensa parcialmente las emisiones de la cría. Además, los cerdos de bellota dependen menos de piensos cultivados (con su huella asociada de fertilizantes y transporte) ya que gran parte de su engorde final proviene de bellotas y pastos naturales \cite{noauthor_huella_2025} \cite{noauthor_producir_nodate}. En cambio, un cerdo ibérico de cebo intensivo, alimentado con pienso en granja, tendrá más emisiones por cultivo de cereales y gestión de purines.

Otro factor relevante es la gestión del estiércol: en tratamientos extensivos, los excrementos se dispersan y pueden fertilizar la dehesa, mientras que en usos intensivso hay riesgo de emisiones de metano y óxido nitroso si no se tratan adecuadamente.

Por su parte, la fase de curación tiene impacto por el consumo eléctrico de los secaderos (aunque muchos secaderos tradicionales usan aire natural de la sierra, las instalaciones modernas sí usan refrigeración/ventilación controlada). No obstante, se estima que esta fase energética es menos determinante que la producción del pienso o la emisión entérica en granja.

Finalmente, la etapa de distribución y logística (transporte del producto final a mercados nacionales e internacionales) contribuye adicionalmente a la huella: si el jamón se exporta por vía marítima o aérea, esas emisiones pueden ser significativas en el balance final \cite{noauthor_huella_2025}. Sin embargo, en peso relativo, el transporte suele ser menor que la producción primaria; por ejemplo, en la DOP Teruel atribuyeron parte de su menor huella comparada a ibéricos a que toda la producción se realiza localmente, minimizando transportes entre fases \cite{noauthor_producir_nodate}.

Comparado con otras proteínas animales, el jamón ibérico tiene un impacto moderado: considerablemente menor que la carne de vacuno (la cual puede triplicar o más las emisiones por kg, debido al metano de los rumiantes), y similar o ligeramente menor que otros productos porcinos intensivos gracias al secuestro de carbono de la dehesa y la ausencia de estabulación continua. Por supuesto, ninguna carne compite en sostenibilidad con las proteínas vegetales, pero dentro del espectro cárnico, el ibérico de bellota aparece como una opción relativamente más sostenible en cuanto a carbono.

Además de emisiones de GEI, la cadena del ibérico conlleva otros impactos: uso de suelo elevado (la dehesa requiere grandes superficies por animal, aunque mantiene un ecosistema valioso), consumo de agua (los cerdos beben y se requieren litros de agua para curado y limpieza, aunque menor que en otras producciones intensivas), e impactos sobre la biodiversidad (la dehesa, si se conserva, favorece la biodiversidad; pero una sobreexplotación podría degradarla).

El perfil ambiental del jamón ibérico está marcado por un contraste: impactos significativos en la cría por ser ganadería (como cualquier producto cárnico), contrarrestados parcialmente por el valor ecológico de la dehesa y prácticas tradicionales extensivas que, bien gestionadas, pueden alinearse con la conservación del entorno \cite{noauthor_huella_2025}.

\subsection{Políticas y documentos sobre sostenibilidad ganadera en la UE y en España}
La preocupación por la sostenibilidad de los productos de origen animal ha ido en aumento en la agenda política europea y nacional. Los documentos estratégicos de UE y España empujan al sector jamonero hacia la sostenibilidad en tres ejes: ambiental (menos emisiones y contaminación, proteger dehesa), social (fijar población rural, buen trato animal) y económico (viabilidad a largo plazo). Esto sienta un marco propicio para que un Pasaporte Digital de Producto del jamón ibérico incluya métricas de sostenibilidad (ej. huella de carbono o hídrica por pieza, porcentaje de materiales reciclados en envase, información sobre bienestar animal certificado), respondiendo a las prioridades marcadas por estas políticas. A nivel de la Unión Europea, el Pacto Verde Europeo y la estrategia \textit{De la Granja a la Mesa} (Farm to Fork, 2020) establecen objetivos claros para reducir el impacto ambiental de la ganadería, incluyendo la reducción de emisiones de metano, uso más eficiente de fertilizantes en cultivos para pienso, bienestar animal y promoción de dietas más sostenibles \cite{noauthor_farm_nodate} \cite{noauthor_sistema_nodate}. Aunque estas estrategias no son normas vinculantes por sí mismas, han impulsado propuestas legislativas (p.ej. el Reglamento de Metano en agricultura en discusión, o la revisión de la PAC con eco-esquemas ganaderos).

En el contexto del jamón ibérico, la dehesa es reconocida como un sistema agrosilvopastoril de alto valor natural, por lo que políticas de desarrollo rural y conservación de la biodiversidad en España prestan especial atención a su mantenimiento. Documentos del Ministerio para la Transición Ecológica y Reto Demográfico (MITERD) identifican a la dehesa como sumidero de carbono y base para sistemas ganaderos sostenibles, sugiriendo incentivos para su preservación. Asimismo, el Ministerio de Agricultura (MAPA) ha publicado guías sobre cálculo de la huella de carbono en explotaciones ganaderas, fomentando que sectores como el porcino ibérico cuantifiquen y reporten sus emisiones. Un ejemplo notable es la iniciativa de la DOP Jamón de Teruel, pionera en España en realizar un estudio completo de ACV para su producto (citado anteriormente), cuyos resultados fueron presentados con apoyo de autoridades locales \cite{noauthor_producir_nodate}. Esto refleja una tendencia general: el propio sector y las administraciones colaboran para estudiar el impacto y mejorar la sostenibilidad (por ejemplo, ASICI podría seguir con un estudio similar para ibéricos en el futuro).

La UE, por su parte, está desarrollando metodologías comunes como el Product Environmental Footprint (PEF) que en un futuro podrían aplicarse a productos cárnicos específicos para comunicar su perfil ambiental de forma normalizada \cite{noauthor_product_nodate}. De hecho, algunos productores ya buscan certificaciones voluntarias: existen jamones ibéricos certificados con la huella de carbono por AENOR, indicando las emisiones por pieza (ej. Jamones Eíriz declaró 28,9 kg CO\textsubscript{2}e por jamón de bellota de 7,8 kg) \cite{burgues_jamones_2015}. Esto se alinea con regulaciones emergentes: el Reglamento (UE) 2024/2065 (sobre alegaciones ecológicas) previsiblemente exigirá verificar con metodologías oficiales cualquier afirmación como <<baja huella de carbono>> del producto \cite{noauthor_directiva_nodate}. En España, se aprobó el Plan Estratégico de la PAC de España, que promueve buenas prácticas ganaderas (alimentación de precisión, gestión de purines para biogás, dehesa como sumidero) para reducir emisiones y adaptarse al cambio climático \cite{noauthor_plan_nodate}. También cobra fuerza la noción de economía circular en ganadería: aprovechar subproductos (estiércol, sueros, huesos) para compost, energía o nuevos productos.

\subsection{Circularidad postconsumo y fin de vida.}
Si bien un jamón ibérico es en sí un producto consumible (no un bien duradero que se <<recicla>>), existen posibilidades de mejora en circularidad asociadas a él. Un enfoque de pasaporte digital permite añadir información y funciones para el fin de vida tanto del producto como de sus envases y residuos. Por ejemplo, el pasaporte podría informar al consumidor sobre cómo aprovechar al máximo el producto (recomendaciones para reutilizar el hueso del jamón en caldos, reciclar la grasa en cocina, etc.), evitando desperdicios alimentarios. De hecho, tradicionalmente el hueso del jamón se emplea para hacer caldo, y la grasa fundida puede reutilizarse culinariamente; fomentar esto reduce residuos orgánicos. En cuanto al envase, muchos jamones se venden deshuesados o loncheados en envases plásticos al vacío. Un DPP podría especificar el material del envoltorio (ej. <<PET reciclable>> o <<bioplástico compostable>>) y dar instrucciones claras de disposición (qué contenedor usar, si tiene componentes separables) \cite{noauthor_digital_nodate-1}. Esta clase de orientación al consumidor encaja con las políticas de economía circular de la UE, que buscan mejorar las tasas de reciclaje. Incluso para el jamón con hueso, que suele venir envuelto en papel encerado o tela, el pasaporte digital puede indicar cómo desechar o compostar esos materiales.Otra faceta de circularidad es la reutilización de subproductos industriales: en la elaboración del jamón, quedan salmuera usada, recortes de carne, etc.

A nivel B2B, un pasaporte digital podría enlazar información sobre cómo esos subproductos han sido reaprovechados (por ejemplo, si la empresa convierte recortes en piensos, o si la sal residual se emplea para deshielo de carreteras). Aunque esto va más allá de lo que normalmente vería un consumidor, sí podría formar parte de indicadores de circularidad en informes de sostenibilidad agregados. Adicionalmente, la prolongación de la vida útil es un principio circular: en jamones curados, la vida útil ya es larga, pero un DPP podría certificar las condiciones óptimas de conservación postventa (temperatura, protección de la pieza empezada) para evitar que el producto se estropee prematuramente.

Cabe mencionar iniciativas en otros sectores alimentarios hacia envases retornables o sistemas de depósito – si en el futuro se implementaran (por ejemplo, envases de loncheados con depósito), el DPP podría registrar cuántos envases fueron devueltos y reciclados. En síntesis, aunque la circularidad en un producto cárnico tiene límites (no es reciclable como un aparato electrónico), sí existen acciones para cerrar el ciclo en lo posible: maximizar el aprovechamiento alimentario, reciclar envases y subproductos, y suministrar información para minimizar la huella postconsumo. El Pasaporte Digital del jamón ibérico puede ser la herramienta para comunicar y ejecutar estas acciones, demostrando compromiso con la economía circular tanto al regulador como al consumidor final.

\section{Análisis de Riesgos y Barreras de Adopción: Tecnológicas, Económicas, Regulatorias}\label{sec:evaluacion-riesgos}

\section{Comparación con Soluciones Alternativas: PKI, Bases de Datos Tradicionales}\label{sec:evaluacion-comparacion}
\subsection{PKI}
\subsection{Soluciones tradicionales (certificados electrónicos, criptografía, bases de datos)}
Antes de adoptar blockchain/DLT, vale la pena preguntarse: ¿por qué no usar simplemente un sistema centralizado convencional para la trazabilidad del jamón? De hecho, el sector ibérico ya dispone del mencionado Sistema ÍTACA, que es esencialmente una base de datos centralizada administrada por ASICI, donde se recogen todos los movimientos (nacimiento, engorde, sacrificio, etc.) de cada cerdo/jamón \cite{noauthor_sistema_nodate} \cite{noauthor_sello_nodate}. Este sistema cumple muy bien con la trazabilidad legal y proporciona transparencia a autoridades. Sin embargo, sistemas centralizados presentan algunas limitaciones:
\begin{itemize}
    \item Dependencia de una entidad central: hay que confiar en ASICI (u otra entidad) para la integridad de los datos. Si hubiera un fallo de seguridad o mala praxis, los datos podrían alterarse sin forma fácil de detectar cambios a posteriori.
    \item Acceso limitado al consumidor: actualmente ÍTACA tiene una app para consumidores, pero en general las bases de datos corporativas no están abiertas al escrutinio público.
    \item Interoperabilidad restringida: un sistema propio puede costar integrarlo con sistemas de otros países u otras cadenas alimentarias, mientras que una DLT estandarizada facilita compartir datos con, por ejemplo, un importador en Japón que también use blockchain.
\end{itemize}

Dicho eso, se podría lograr mucho con sistemas tradicionales potentes: un esquema de certificación electrónica donde cada operador firma digitalmente (PKI) los documentos de trazabilidad de un lote, y esos certificados se almacenan en un repositorio central. Esto daría autenticidad (cada registro firmado por quien corresponde) y cierto grado de inmutabilidad (no se puede alterar un registro firmado sin invalidar la firma); de hecho, muchas cadenas de suministro funcionan así, usando tecnologías de seguridad comunes (TLS, firmas X.509, etc.). La diferencia clave de una DLT es que logra inmutabilidad y consenso global sin depender de la autoridad de certificación central, es decir, las reglas de integridad están aseguradas por el protocolo y los nodos distribuídos, no por la firma de una entidad. En un entorno como el ibérico donde ASICI, de hecho, agrupa a casi todos los actores, se podría argumentar que “ya existe un tercero de confianza, no hace falta blockchain”.

No obstante, pensemos en casos de uso ampliados: integrar al consumidor final o al importador extranjero como parte interesada que quiere verificar datos sin tener que \textit{creer} ciegamente en los datos proporcionados por la industria española. Una blockchain pública permitiría esa verificación independiente (e.g. un importador podría correr su propio nodo y auditar los movimientos registrados de cierto jamón).

Otro punto: la resiliencia. Un sistema central puede caerse o ser atacado; una red descentralizada es más resiliente, asegurando disponibilidad de la información incluso si un nodo (o servidor de ASICI) falla.Desde luego, lo tradicional suele ser más sencillo y conocido: bases de datos SQL, servidores web, etc., con los que hay gran experiencia. Blockchain implica adquirir nuevas habilidades y puede ser costoso de desarrollar inicialmente. Por eso, en muchos casos se opta por soluciones híbridas: mantener una base de datos central robusta (para operaciones diarias y velocidad), pero anclar o replicar ciertos datos en una blockchain para ganar inmutabilidad y transparencia.

Esta podría ser una vía para el jamón ibérico: ÍTACA podría seguir siendo el sistema maestro interno, pero con una capa blockchain alimentada con eventos hash o resúmenes para que terceros validen. En cuanto a criptografía pura vs blockchain: podríamos imaginar un esquema donde cada evento de trazabilidad se firma digitalmente y se envía al siguiente actor, que verifica la cadena de firmas (similar a cómo funcionan algunos sistemas de EDI). Esto da seguridad punto a punto, pero no provee una vista única compartida de todos los eventos ordenados en el tiempo; cada parte solo ve su entrada y salida, a menos que se cree un repositorio común. Blockchain precisamente crea ese \textit{libro mayor distribuido} que todos comparten y aceptan.

A modo de resumen, las soluciones tradicionales funcionan pero no logran por sí solas el nivel de confianza distribuida y transparencia pública que ofrece una DLT. El Pasaporte Digital de Producto, tal como lo concibe la UE, parece orientado a aprovechar esa confianza distribuida: se espera que la información del DPP esté en un “registro digital seguro” accesible por múltiples partes \cite{motusic_understanding_2024}. Esto sugiere alguna forma de blockchain o infraestructura descentralizada. Por lo tanto, implementar un DPP de jamón ibérico con DLT puede dar un salto cualitativo en garantía de integridad frente a simplemente colgar un PDF en una web (que podría alterarse fácilmente). Dicho de otra forma: un PDF subido al sitio de una empresa es un \textit{claim} unidireccional, mientras que un registro blockchain es una \textit{prueba colectiva} \cite{motusic_understanding_2024}.